
\documentclass[11pt,a4paper]{article}

% --- Encodage & police ---
\usepackage[T1]{fontenc}
\usepackage[utf8]{inputenc}
\usepackage{tcolorbox}
\usepackage{lmodern}
\usepackage{url}


% --- Mise en page ---
\usepackage[margin=1.5cm]{geometry}

% --- Couleurs ---
\usepackage{xcolor}

% --- Listings pour le code ---
\usepackage{listings}
\lstdefinelanguage{bash}{
  sensitive=true,
  morecomment=[l]{\#},
  morestring=[b]",
}
\lstset{
  language=bash,
  basicstyle=\ttfamily\small,
  numbers=left,
  numberstyle=\scriptsize\color{gray},
  stepnumber=1,
  numbersep=8pt,
  showstringspaces=false,
  columns=fullflexible,
  keepspaces=true,
  tabsize=2,
  breaklines=true,
  frame=single,
  rulecolor=\color{black!20},
  keywordstyle=\bfseries\color{black},
  commentstyle=\itshape\color{gray!70},
  stringstyle=\color{teal!70!black},
}

\title{Mac: Installer GLPI dans un conteneur Docker et un hyperviseur}
\author{}
\date{}

\begin{document}
\maketitle


\section{Cours : Installer un hyperviseur sur Mac et déployer GLPI via Docker}

\tcbset{colback=blue!5!white,colframe=blue!75!black,fonttitle=\bfseries}

\begin{tcolorbox}[title={Introduction}]
Les utilisateurs de Mac peuvent aussi pratiquer la virtualisation et l’installation de GLPI pour superviser des machines virtuelles.  
Cependant, les outils disponibles diffèrent selon que le Mac soit équipé d’un processeur \textbf{Intel} ou \textbf{Apple Silicon (M1/M2/M3)}.  
Dans ce cours, nous allons voir comment installer un hyperviseur adapté, puis comment déployer GLPI via Docker pour centraliser la supervision des VMs.
\end{tcolorbox}

\begin{tcolorbox}[title={1. Choix de l’hyperviseur selon le type de Mac}]
\begin{tabular}{|l|l|l|l|}
\hline
\textbf{Type de Mac} & \textbf{Hyperviseur recommandé} & \textbf{Avantages} & \textbf{Inconvénients} \\
\hline
Mac Intel & VMware Fusion Player (gratuit) & Compatible VMware, stable & Nécessite Intel, pas ARM \\
          & VirtualBox (gratuit)           & Simple, gratuit, multi-OS & Moins optimisé que Fusion \\
\hline
Mac Apple Silicon & UTM (gratuit)            & Compatible ARM, open-source & Performances moyennes en x86 \\
                  & Parallels Desktop (payant) & Simple, performant & Licence payante \\
                  & VMware Fusion Tech Preview & Support ARM & Encore en version test \\
\hline
\end{tabular}

\medskip
\textbf{Conclusion :}  
\begin{itemize}
  \item Sur Mac Intel → privilégier \textbf{VMware Fusion Player}.  
  \item Sur Mac M1/M2/M3 → privilégier \textbf{UTM} (gratuit) ou \textbf{Docker Desktop} si l’objectif principal est GLPI.  
\end{itemize}
\end{tcolorbox}

\begin{tcolorbox}[title={2. Installation d’un hyperviseur sur Mac}]
\begin{enumerate}
  \item Télécharger l’hyperviseur choisi :  
    \begin{itemize}
      \item VMware Fusion Player : \url{https://customerconnect.vmware.com/}  
      \item VirtualBox : \url{https://www.virtualbox.org/}  
      \item UTM : \url{https://mac.getutm.app/}  
    \end{itemize}
  \item Installer l’application en suivant l’assistant d’installation macOS.  
  \item Créer une machine virtuelle avec Ubuntu/Debian (2 Go RAM, 20 Go disque, mode réseau Bridged recommandé).  
\end{enumerate}
\end{tcolorbox}

\begin{tcolorbox}[title={3. Installation de Docker Desktop}]
Docker permet de déployer GLPI sans avoir à installer manuellement Apache, PHP et MariaDB.  
\begin{enumerate}
  \item Télécharger Docker Desktop pour Mac (Intel ou Apple Silicon) : \url{https://www.docker.com/products/docker-desktop/}.  
  \item Installer et lancer Docker Desktop.  
  \item Vérifier l’installation :
\begin{verbatim}
docker --version
\end{verbatim}
\end{enumerate}
\end{tcolorbox}

\begin{tcolorbox}[title={4. Déploiement de GLPI via Docker Compose}]
\begin{enumerate}
  \item Créer un dossier projet :
\begin{verbatim}
mkdir ~/glpi-docker
cd ~/glpi-docker
\end{verbatim}

  \item Créer un fichier \texttt{docker-compose.yml} :
\begin{verbatim}
version: '3.3'
services:
  glpi:
    image: diouxx/glpi
    container_name: glpi
    ports:
      - "8080:80"
    environment:
      - TZ=Europe/Paris
    volumes:
      - ./glpi:/var/www/html/glpi

  db:
    image: mariadb:10.5
    container_name: glpi-db
    restart: always
    environment:
      MYSQL_ROOT_PASSWORD: rootpass
      MYSQL_DATABASE: glpi
      MYSQL_USER: glpi
      MYSQL_PASSWORD: glpipass
    volumes:
      - ./db:/var/lib/mysql
\end{verbatim}

  \item Lancer les conteneurs :
\begin{verbatim}
docker-compose up -d
\end{verbatim}

  \item Vérifier :
\begin{verbatim}
docker ps
\end{verbatim}
\end{enumerate}
\end{tcolorbox}

\begin{tcolorbox}[title={5. Accéder à GLPI}]
\begin{itemize}
  \item Ouvrir le navigateur : \url{http://localhost:8080}  
  \item Suivre l’assistant d’installation :  
    \begin{itemize}
      \item Choisir la langue.  
      \item Base de données → hôte : \texttt{db}, utilisateur : \texttt{glpi}, mot de passe : \texttt{glpipass}, base : \texttt{glpi}.  
    \end{itemize}
  \item Une fois installé, l’interface GLPI est accessible et prête à recevoir des machines supervisées.
\end{itemize}
\end{tcolorbox}

\begin{tcolorbox}[title={6. Supervision des VMs avec l’agent GLPI}]
Sur chaque VM Ubuntu/Debian créée dans l’hyperviseur, installer l’agent GLPI :  
\begin{verbatim}
wget https://github.com/glpi-project/glpi-agent/releases/download/1.9/glpi-agent_1.9-1_all.deb
sudo dpkg -i glpi-agent_1.9-1_all.deb
\end{verbatim}

Configurer l’agent pour pointer vers le serveur GLPI (hôte Mac) :  
\begin{verbatim}
sudo nano /etc/glpi-agent/conf.d/server.conf
\end{verbatim}

Ajouter :
\begin{verbatim}
server = http://IP_MAC:8080/glpi
\end{verbatim}

Puis redémarrer le service :
\begin{verbatim}
sudo systemctl restart glpi-agent
\end{verbatim}
\end{tcolorbox}

\begin{tcolorbox}[title={Conclusion}]
Avec un Mac, il est possible de créer des environnements de virtualisation grâce à des hyperviseurs comme VMware Fusion, VirtualBox ou UTM.  
En parallèle, l’utilisation de Docker Desktop simplifie le déploiement de GLPI.  
Ainsi, les élèves peuvent mettre en place une infrastructure pédagogique où leur Mac héberge GLPI dans Docker, et leurs VMs (Ubuntu/Debian) sont supervisées via l’agent GLPI.
\end{tcolorbox}


\end{document}