

\documentclass[11pt,a4paper]{article}

% --- Encodage & police ---
\usepackage[T1]{fontenc}
\usepackage[utf8]{inputenc}
\usepackage{tcolorbox}
\usepackage{lmodern}
\usepackage{url}


% --- Mise en page ---
\usepackage[margin=1.5cm]{geometry}

% --- Couleurs ---
\usepackage{xcolor}

% --- Listings pour le code ---
\usepackage{listings}
\lstdefinelanguage{bash}{
  sensitive=true,
  morecomment=[l]{\#},
  morestring=[b]",
}
\lstset{
  language=bash,
  basicstyle=\ttfamily\small,
  numbers=left,
  numberstyle=\scriptsize\color{gray},
  stepnumber=1,
  numbersep=8pt,
  showstringspaces=false,
  columns=fullflexible,
  keepspaces=true,
  tabsize=2,
  breaklines=true,
  frame=single,
  rulecolor=\color{black!20},
  keywordstyle=\bfseries\color{black},
  commentstyle=\itshape\color{gray!70},
  stringstyle=\color{teal!70!black},
}

\title{Mac et VM : Communication }
\author{}
\date{}

\begin{document}
\maketitle









\section*{Cours : Communication entre un Mac (hôte) et une VM Ubuntu en mode Bridged}

\tcbset{colback=orange!5!white,colframe=orange!75!black,fonttitle=\bfseries}

\begin{tcolorbox}[title={Introduction}]
Lorsqu’une machine virtuelle est configurée en mode \textbf{Bridged}, elle se comporte comme un ordinateur à part entière connecté au même réseau local que l’hôte (ici ton Mac).  
Cela permet d’établir une communication directe entre le Mac et la VM Ubuntu, comme si deux ordinateurs distincts étaient branchés sur la même box ou le même switch.
\end{tcolorbox}

\begin{tcolorbox}[title={1. Comprendre le mode Bridged}]
\begin{itemize}
  \item Le mode \textbf{NAT} isole la VM derrière l’hôte : elle accède à Internet, mais l’hôte et la VM ne communiquent pas directement.  
  \item Le mode \textbf{Host-only} crée un réseau privé entre la VM et l’hôte, mais sans accès Internet.  
  \item Le mode \textbf{Bridged} connecte la VM directement au réseau physique de l’hôte. Elle obtient une adresse IP du routeur (ex. box Internet) comme n’importe quel autre appareil.  
\end{itemize}
Ainsi, ton Mac et ta VM partagent le même réseau, ce qui autorise le ping, SSH, transferts de fichiers, etc.
\end{tcolorbox}

\begin{tcolorbox}[title={2. Vérification de l’adresse IP de la VM}]
Sur ta VM Ubuntu :  
\begin{verbatim}
ip a
\end{verbatim}
Tu obtiens une adresse IP (exemple : 192.168.1.50) sur l’interface \texttt{ens33}.  
C’est cette adresse que ton Mac utilisera pour communiquer avec la VM.

Vérifie aussi que la VM a Internet :  
\begin{verbatim}
ping -c 4 8.8.8.8
ping -c 4 google.com
\end{verbatim}
\end{tcolorbox}

\begin{tcolorbox}[title={3. Tester la communication depuis le Mac}]
Sur ton Mac, ouvre un terminal et exécute :  
\begin{verbatim}
ping 192.168.1.50
\end{verbatim}
(En remplaçant par l’adresse IP réelle de ta VM).  
Si tu reçois des réponses, la communication réseau est fonctionnelle.

\textbf{Astuce} : Pour connaître l’adresse IP de ton Mac et vérifier qu’il est dans le même sous-réseau :  
\begin{verbatim}
ifconfig
\end{verbatim}
Exemple : ton Mac est en 192.168.1.20 et ta VM en 192.168.1.50 → ils peuvent communiquer.
\end{tcolorbox}

\begin{tcolorbox}[title={4. Activer SSH pour accéder à la VM depuis le Mac}]
Sur ta VM Ubuntu, installe et active OpenSSH :  
\begin{verbatim}
sudo apt update
sudo apt install openssh-server -y
sudo systemctl enable ssh
sudo systemctl start ssh
\end{verbatim}

Vérifie que le service tourne :  
\begin{verbatim}
sudo systemctl status ssh
\end{verbatim}

Ensuite, depuis ton Mac :  
\begin{verbatim}
ssh utilisateur@192.168.1.50
\end{verbatim}
(Remplace \texttt{utilisateur} par ton compte Ubuntu).  
Tu auras alors un accès terminal direct à la VM depuis ton Mac.
\end{tcolorbox}

\begin{tcolorbox}[title={5. Transférer des fichiers entre Mac et VM}]
Depuis ton Mac, tu peux transférer un fichier vers la VM avec :  
\begin{verbatim}
scp fichier.txt utilisateur@192.168.1.50:/home/utilisateur/
\end{verbatim}

Et récupérer un fichier de la VM :  
\begin{verbatim}
scp utilisateur@192.168.1.50:/home/utilisateur/fichier.txt .
\end{verbatim}
Cela utilise le protocole SSH pour sécuriser les transferts.
\end{tcolorbox}

\begin{tcolorbox}[title={6. Points à vérifier en cas de problème}]
\begin{itemize}
  \item Vérifier que la VM et le Mac sont dans le même sous-réseau (192.168.1.x).  
  \item Désactiver temporairement le pare-feu Ubuntu si besoin :
\begin{verbatim}
sudo ufw disable
\end{verbatim}
  \item Vérifier que le routeur/box autorise le bridge sur le Wi-Fi (certaines box bloquent le bridging sur Wi-Fi). Dans ce cas, forcer VMware à utiliser l’interface Wi-Fi dans les paramètres réseau avancés.
\end{itemize}
\end{tcolorbox}

\begin{tcolorbox}[title={Conclusion}]
En mode Bridged, ta VM Ubuntu devient un hôte du réseau local à part entière.  
Depuis ton Mac, tu peux la joindre par son IP, la superviser avec GLPI, te connecter en SSH et transférer des fichiers.  
Cette configuration est idéale pour des scénarios de supervision et d’administration réseau car elle reproduit le comportement d’une vraie machine physique connectée au LAN.
\end{tcolorbox}



\end{document}