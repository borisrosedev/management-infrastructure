\documentclass[12pt, letterpaper]{article}
\usepackage{graphicx}
\usepackage[most]{tcolorbox}
\usepackage{minted}
\usepackage{hyperref}
\usepackage[margin=1.5cm]{geometry}

\hypersetup{
    colorlinks=true,
    linkcolor=blue,
    filecolor=magenta,      
    urlcolor=cyan,
    pdftitle={Travaux Pratiques},
    pdfpagemode=FullScreen,
    }
\graphicspath{{images/}} 

\title{TP 1 bis : Gestion des incidents }
\author{Formateur: Boris Rose}
\date{\today}


\begin{document}

\maketitle

\section*{Introduction}

\begin{tcolorbox}[colback=brown!5,colframe=brown!60!black,title=Consignes]

\begin{itemize}
    \item Créer un dépôt sur github ou gitlab si ce n'est pas déjà fait (voir la vidéo dans le dossier google drive que j'ai partagé)
    \item Dans ce dépôt que vous aurez créé, faire les différents tps dans différentes branches
    \item Chaque branche doit commencer par feature/ suivi du nom du tp 
\end{itemize}

\end{tcolorbox}


\section*{Consignes}

\begin{tcolorbox}[colback=cyan!5,colframe=cyan!60!black,title=Figma d'étude]
    Vous répondrez aux questions en vous aidant du cours ( voir Drive ) et du lien vers le Figma du scénario suivant: 
    \url{https://www.figma.com/board/7tc1gkdVbMGXBPO6ucIDCu/Sc\%C3\%A9narios?node-id=988-3446&t=eyBtpBONZV4mkin3-1}
\end{tcolorbox}


\section*{Partie 0: Installer GLPI}
\paragraph{\color{red}{Si vous n'avez pas déjà fait cela dans le TP1}}

L'idée est de suivre ce qui est dans le cours, dans le dossier config:
\begin{itemize}
    \item mac-glpi-hyperviseur.pdf 
    \item mac-vm-communication.pdf
    \item vm-ubuntu.pdf
\end{itemize}

\begin{tcolorbox}[colback=orange!5,colframe=orange!60!black,title=Attention]
Si pour d'autres matières vous avez déjà des machines virtuelles managées par un hyperviseur (T2) tel que VMWorstation ou VirtualBox 
installez les agents GLPI dans ces MV. Autrement dit vous n'avez pas besoin d'installer une VM avec OS Ubuntu.
\end{tcolorbox}

\section*{Partie 1 : Gestion des incidents}


\begin{tcolorbox}[colback=brown!5,colframe=brown!60!black,title=Définition]
    La gestion des incidents est le processus permettant de détecter, enregistrer, diagnostiquer et
résoudre les pannes informatiques.
Un incident est un événement imprévu perturbant le fonctionnement normal d’un service
informatique 
\end{tcolorbox}

\begin{itemize}
    \item Quel incident peut concerner le serveur web de l'entreprise sur le diagramme Figma ?
    \item Donner la liste des logiciels critiques installés sur le disque ssd du serveur physique auquel avait trait la question précédente ?
    \item Quelle est la relation entre GLPI et cet incident ?
    \item Quel logiciel peut jouer le rôle d'intermédiaire entre GLPI et l'incident ?
    \item Quelles seraient les bonnes pratiques ITIL par rapport à la gestion de cet incident ?
\end{itemize}


\section*{Partie 2 : Supervision et monitoring}


\begin{tcolorbox}[colback=brown!5,colframe=brown!60!black,title=Définition]
La supervision permet de surveiller en temps réel l’état du parc (CPU, mémoire, réseau,
disques).
\end{tcolorbox}



\begin{itemize}
    \item Quels sont les outils que vous auriez pu combiner ( voir le cours ) pour faire une supervision réseau et avoir une visualisation des métriques de performance
\end{itemize}



\section*{Partie 3 : Sécurité du parc informatique}


\begin{tcolorbox}[colback=brown!5,colframe=brown!60!black,title=Définition]
La sécurité du parc regroupe les mesures techniques et organisationnelles pour protéger le
matériel et les données.
\end{tcolorbox}

\begin{itemize}
    \item Quelles sont les mesures prises ( dans le scénario du Figma ) pour sécuriser le parc informatique ?
\end{itemize}


\section*{Partie 4 : Demandes utilisateur}

\begin{tcolorbox}[colback=green!5,colframe=green!60!black,title=Figme d'étude (difficile)]
Observer le scénario suivant pour répondre aux question: 
\url{https://www.figma.com/board/7tc1gkdVbMGXBPO6ucIDCu/Sc%C3%A9narios?node-id=707-892&t=eyBtpBONZV4mkin3-1}
\end{tcolorbox}


\begin{tcolorbox}[colback=brown!5,colframe=brown!60!black,title=Définition]
Une demande utilisateur est une requête planifiée ou standard, souvent récurrente et docu-
mentée. Contrairement à un incident, elle ne traduit pas une panne mais un besoin d’assistance
ou d’accès.
\end{tcolorbox}

\begin{itemize}
    \item Trouvez dans le scénario Figma la demande utilisateur ?  
    \item Quel est le cycle de vie d’une demande utilisateur ( voir le cours )
\end{itemize}

\end{document}





