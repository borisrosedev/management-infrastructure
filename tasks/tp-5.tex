\documentclass[12pt, letterpaper]{article}
\usepackage{graphicx}
\usepackage[most]{tcolorbox}
\usepackage{minted}
\usepackage{hyperref}
\usepackage[margin=1.5cm]{geometry}
\usepackage{xcolor} % to access the named colour LightGray
\definecolor{LightGray}{gray}{0.9}

\hypersetup{
    colorlinks=true,
    linkcolor=blue,
    filecolor=magenta,      
    urlcolor=cyan,
    pdftitle={Travaux Pratiques},
    pdfpagemode=FullScreen,
    }
\graphicspath{{images/}} 

\title{TP 5}
\author{Formateur: Boris Rose}
\date{}


\begin{document}

\maketitle

\section{Objectifs pédagogiques}

\begin{tcolorbox}[colback=cyan!5,colframe=cyan!60!black,title=Objectifs]
   \begin{itemize}
    \item Comprendre les éléments de configuration de GLPI 
    \item Virtualiser un parc informatique 
    \item Permettre une communication peer-to-peer sur un réseau privé
   \end{itemize}
\end{tcolorbox}


\section{Consignes}
\begin{tcolorbox}[colback=cyan!5,colframe=cyan!60!black,title=Consignes]
   \begin{itemize}
    \item 
   \end{itemize}
\end{tcolorbox}




\section{Figma}

\begin{tcolorbox}[colback=cyan!5,colframe=cyan!60!black,title=Figma du Cours]
    \url{https://www.figma.com/board/7tc1gkdVbMGXBPO6ucIDCu/Sc%C3%A9narios?node-id=707-892&t=rUb3lOtsKM8l73vp-1}
\end{tcolorbox}







\section{VM}

\begin{itemize}
    \item Qu'est qu'ISO Ubuntu ?
    \item Que signifie Live Session dans le cas d'ISO Ubuntu ?
    \item Qu'est-ce que la RAM ?
    \item Qu'est-ce que le VM Disk ?
    \item Que sous-entend la notion de systèmes d'exploitation invités ?
    \item \textit{Chaque machine virtuelle croit disposer de ses propres ressources matérielles ...} Que sous-entend cette phrase ?
    \item Quel est le rôle de l'hyperviseur ?
\end{itemize}


\section{Locale}

\begin{itemize}
    \item Que signifient les commandes suivantes ?
\end{itemize}
\begin{minted}{bash}
    locale
    localectl status.
\end{minted}

\section{Netplan}

\begin{itemize}
    \item Qu'est-ce que netplan pour Ubuntu ? (voir l'extrait de code ci-dessous)
    \item Qu'est-ce qu'est-ce que le service DHCP ?
\end{itemize}

\begin{minted}{yaml}
#sudo nano /etc/netplan/00-installer-config.yaml

network:
  version: 2
  renderer: networkd
  ethernets:
    ens33:
      dhcp4: true
    
\end{minted}


\section{SSH}

\begin{minted}{bash}

    sudo apt update
    sudo apt install openssh-server -y
    sudo systemctl enable ssh
    sudo systemctl start ssh
    
\end{minted}
\begin{itemize}
    \item À quoi servent les commandes ci-dessus ? 
\end{itemize}


\section{ICMP et DNS}

\begin{itemize}
    \item Qu'est-ce que le protocole ICMP 
    \item Qu'est-ce qu'un serveur DNS 
    \item Schématiser le fonctionnement du réseau DNS
    \item Comment envoyer 4 paquets ICMP vers le serveur DNS de Google ?
\end{itemize}


\section{Vérification du disque et du système}

\begin{itemize}
    \item Quelle est la commande sous Ubuntu qui permet de  lister les disques et partitions ?
    \item Celle qui permet de vérifier l’espace disque utilisé ?
    \item Celle qui permet de cérifier la mémoire RAM ?
\end{itemize}



\section{Docker}


\begin{itemize}
    \item Qu'est-ce que Docker 
    \item Comparer Docker avec la Virtualisation Machine
    \item Combien de conteneurs vais-je créer en demandant à Docker de lire le fichier de configuration ci-dessous ?
\end{itemize}




\begin{minted}{yaml}


version: '3.3'
services:
  glpi:
    image: diouxx/glpi
    container_name: glpi
    ports:
      - "8080:80"
    environment:
      - TZ=Europe/Paris
    volumes:
      - ./glpi:/var/www/html/glpi

  db:
    image: mariadb:10.5
    container_name: glpi-db
    restart: always
    environment:
      MYSQL_ROOT_PASSWORD: caroline123
      MYSQL_DATABASE: glpi
      MYSQL_USER: glpi
      MYSQL_PASSWORD: caroline123
    volumes:
      - ./db:/var/lib/mysql


    
\end{minted}


\begin{itemize}
    \item Que font les commandes suivantes ? Où est-ce qu'on doit les exécuter à votre avis ? Sur l'hôte Windows en sachant qu'il s'agit du serveur GLPI ou sur la VM objet de la supervision ?
\end{itemize}

\begin{minted}{bash}
    
    wget https://github.com/glpi-project/glpi-agent/releases/download/1.9/glpi-agent_1.9-1_all.deb
    
    sudo dpkg -i glpi-agent_1.9-1_all.deb

    sudo nano /etc/glpi-agent/conf.d/server.conf
    # in the server.conf add this % server = http://IP_HOST:8080/glpi 
    # where IP_HOST is replaced by the true server ip address
    sudo systemctl restart glpi-agent
    
\end{minted}



\end{document}