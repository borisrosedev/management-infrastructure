\documentclass[12pt, letterpaper]{article}
\usepackage{graphicx}
\usepackage[most]{tcolorbox}
\usepackage{minted}
\usepackage{hyperref}
\usepackage[margin=1.5cm]{geometry}
\usepackage{xcolor} % to access the named colour LightGray
\definecolor{LightGray}{gray}{0.9}

\hypersetup{
    colorlinks=true,
    linkcolor=blue,
    filecolor=magenta,      
    urlcolor=cyan,
    pdftitle={Travaux Pratiques},
    pdfpagemode=FullScreen,
    }
\graphicspath{{images/}} 

\title{TP 2}
\author{Formateur: Boris Rose}
\date{\today}


\begin{document}

\maketitle

\section*{Introduction}

\begin{tcolorbox}[colback=brown!5,colframe=brown!60!black,title=Consignes]

\begin{itemize}
    \item Créer un dépôt sur github ou gitlab si ce n'est pas déjà fait (voir la vidéo dans le dossier google drive que j'ai partagé)
    \item Dans ce dépôt que vous aurez créé, faire les différents tps dans différentes branches
    \item Chaque branche doit commencer par feature/ suivi du nom du tp 
\end{itemize}


\end{tcolorbox}


\section*{Partie 1}


\begin{tcolorbox}[colback=cyan!5,colframe=cyan!60!black,title=Figma du Cours]
    \url{https://www.figma.com/board/7tc1gkdVbMGXBPO6ucIDCu/Sc%C3%A9narios?node-id=707-892&t=rUb3lOtsKM8l73vp-1}
\end{tcolorbox}


\begin{tcolorbox}[colback=brown!5,colframe=brown!60!black,title=Supervision]
    La supervision (Zabbix, Nagios) sert à surveiller l’état des systèmes et réseaux. Lorsqu’une alerte est détectée (panne, surcharge), on peut automatiser la création d’un ticket GLPI. Ainsi, le support est averti immédiatement et le suivi est tracé.
\end{tcolorbox}

\begin{itemize}
    \item En observant le diagramme Figma, combien de ticket GLPI d'incidents ont été créés via une action Zabbix ?
\end{itemize}

\begin{tcolorbox}[colback=brown!5,colframe=brown!60!black,title=Sauvegarde]
Une sauvegarde simple n’est pas suffisante en production. 
\end{tcolorbox}

\begin{itemize}
    \item En observant le diagramme Figma et plus particulièrement le code sh lié à Francis, expliquez ce qu'il a mis en place pour s'assurer de la conservation sécurisée des données GLPI 
    \item En observant dans le dépôt github du cours, le dossier database et plus particulièrement seed.sql, expliquez chaque ligne des instructions sql qui permettent la création de la base de données GLPI (vous pouvez vous aider du cours que j'ai mis dans le dossier docs sur mysql)
\end{itemize}


\section{Partie 2}

\begin{minted}[
frame=lines,
framesep=2mm,
baselinestretch=1.2,
bgcolor=LightGray,
fontsize=\footnotesize,
linenos
]{bash}
#!/usr/bin/env bash
set -Eeuo pipefail
IFS=$'\n\t'

WORKING_DIR="$(cd -- "$(dirname -- "${BASH_SOURCE[0]}")" && pwd -P)"  
\end{minted}

\begin{tcolorbox}
    Il s'agit d'un extrait du init.sh. 
\end{tcolorbox}

\begin{itemize}
    \item Expliquer la première ligne
    \item Expliquer la deuxième ligne (tester l'efficacité de chaque option)
    \item Expliquer la troisième ligne 
    \item Expliquer la cinquième ligne
\end{itemize}


\end{document}