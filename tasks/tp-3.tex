\documentclass[12pt, letterpaper]{article}
\usepackage{graphicx}
\usepackage[most]{tcolorbox}
\usepackage{minted}
\usepackage{hyperref}
\usepackage[margin=1.5cm]{geometry}
\usepackage{xcolor} % to access the named colour LightGray
\definecolor{LightGray}{gray}{0.9}

\hypersetup{
    colorlinks=true,
    linkcolor=blue,
    filecolor=magenta,      
    urlcolor=cyan,
    pdftitle={Travaux Pratiques},
    pdfpagemode=FullScreen,
    }
\graphicspath{{images/}} 

\title{TPs}
\author{Formateur: Boris Rose}
\date{\today}


\begin{document}

\maketitle

\section*{Introduction}

\begin{tcolorbox}[colback=brown!5,colframe=brown!60!black,title=Consignes]

\begin{itemize}
    \item Créer un dépôt sur github ou gitlab si ce n'est pas déjà fait (voir la vidéo dans le dossier google drive que j'ai partagé)
    \item Dans ce dépôt que vous aurez créé, faire les différents tps dans différentes branches
    \item Chaque branche doit commencer par feature/ suivi du nom du tp 
    \item Toute utilisation inutile d'IA sera sanctionné (autrement dit, vous avez le droit de vous aider de l'IA pour comprendre à termes ce que vous écrivez, si vous vous êtes aidé(e) de l'IA mais qu'au final vous n'avez rien compris, je ne mettrai pas de note à votre TP , la note du contrôle DST sera la seule à compter)
\end{itemize}


\end{tcolorbox}


\section*{TP 5}


\begin{tcolorbox}[colback=cyan!5,colframe=cyan!60!black,title=Figma du Cours]
    \url{https://www.figma.com/board/7tc1gkdVbMGXBPO6ucIDCu/Sc%C3%A9narios?node-id=707-892&t=rUb3lOtsKM8l73vp-1}
\end{tcolorbox}

\begin{itemize}
    \item Quels sont les avantages d'une bonne gestion des incidents (observer le diagramme)

\end{itemize}

\begin{itemize}
    \item En quoi consiste la gestion budgétaire du parc informatique ? 
    \item Déterminer en fonction du scénario (diagramme) ce qui va rentrer dans les OpEx et ce qui va rentrer dans les CapEx ?
    \item Faire un budget prévisionnel (au regard du diagramme)
    \item Qu'est-ce que le TCO ?
    \item Qu'est-ce que le ROI ?
    \item Qu'est-ce que le shelfware ?
    \item Qu'est-ce que l'infogérance
    \item Qu'est-ce qu'un SaaS ?
\end{itemize}



\section{TP 6}

\begin{tcolorbox}
    Un serveur coûte 10 000 € à l’achat, 2 000 €/an de maintenance, 1 000 €/an d’énergie
et dure 5 an
    \begin{itemize}
        \item Ecrire une fonction en bash ou powershell(.NET) qui va déterminer le TCO du serveur (cette fonction doit être réutilisable pour d'autres problèmes que celui ci-dessus)
    \end{itemize}
\end{tcolorbox}

\begin{tcolorbox}
    Une entreprise investit 25 000 € dans un outil de supervision.  
Cet outil permet d’économiser 40 000 € par an en temps de panne évité.  
 \begin{itemize}
        \item Ecrire une fonction en bash ou powershell(.NET) qui va déterminer le ROI de l'outil au bout d'un an (cette fonction doit être réutilisable pour d'autres problèmes que celui ci-dessus)
    \end{itemize}
\end{tcolorbox}

\begin{tcolorbox}
    Un parc de 50 ordinateurs est acheté pour un coût total de 60 000 €. La durée de vie prévue est
de 5 ans.
\begin{itemize}
    \item Calculer l’amortissement annuel par ordinateur.
\end{itemize}
\end{tcolorbox}


\end{document}