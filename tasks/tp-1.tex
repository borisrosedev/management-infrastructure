\documentclass[12pt, letterpaper]{article}
\usepackage{graphicx}
\usepackage[most]{tcolorbox}
\usepackage{minted}
\usepackage{hyperref}
\usepackage[margin=1.5cm]{geometry}

\hypersetup{
    colorlinks=true,
    linkcolor=blue,
    filecolor=magenta,      
    urlcolor=cyan,
    pdftitle={Travaux Pratiques},
    pdfpagemode=FullScreen,
    }
\graphicspath{{images/}} 

\title{TP 1}
\author{Formateur: Boris Rose}
\date{\today}


\begin{document}

\maketitle

\section*{Introduction}

\begin{tcolorbox}[colback=brown!5,colframe=brown!60!black,title=Consignes]

\begin{itemize}
    \item Créer un dépôt sur github ou gitlab si ce n'est pas déjà fait (voir la vidéo dans le dossier google drive que j'ai partagé)
    \item Dans ce dépôt que vous aurez créé, faire les différents tps dans différentes branches
    \item Chaque branche doit commencer par feature/ suivi du nom du tp 
\end{itemize}


\end{tcolorbox}


\section*{TP 1}

\begin{tcolorbox}[colback=cyan!5,colframe=cyan!60!black,title=Définition]
La gestion du parc informatique désigne l’ensemble des pratiques, processus et outils permettant de superviser, sécuriser et maintenir les ressources matérielles (ordinateurs, serveurs, imprimantes, périphériques) et logicielles (systèmes d’exploitation, applications, licences) d’une organisation.
\end{tcolorbox}

\begin{itemize}
    \item Qu'est-ce qu'un périphérique ?
    \item Qu'est-ce qu'un serveur
    \item Qu'est-ce qu'un système d'exploitation ?
\end{itemize}


\begin{tcolorbox}[colback=cyan!5,colframe=cyan!60!black,title=Figma du Cours]
    \url{https://www.figma.com/board/7tc1gkdVbMGXBPO6ucIDCu/Sc%C3%A9narios?node-id=707-892&t=rUb3lOtsKM8l73vp-1}
\end{tcolorbox}


\begin{itemize}
    \item Dans quelle mesure l'organisation TeaRoom gère correctement le périphérique bidirectionnel indiqué dans le diagramme FigJamBoard ?
\end{itemize}


\begin{itemize}
    \item Quel est l'intérêt de faire un inventaire ?
    \item En quoi consiste une "gestion proactive" ? (page 5 du cours)
    \item Que signifie "vérifier le respect des licences logicielles" ?
    \item Qu'est-ce que le shadow IT ? (contexte: voir dépôt github du cours / docs / fondamentaux.pdf )
\end{itemize}


\section*{TP 2}

\begin{tcolorbox}[colback=cyan!5,colframe=cyan!60!black,title=Définition]
La gestion des incidents est le processus qui vise à identifier, enregistrer, analyser et résoudre les problèmes techniques affectant le parc informatique. Son objectif principal est de minimiser l’impact sur les utilisateurs et sur l’activité de l’organisation.
\end{tcolorbox}

\begin{itemize}
    \item En observant le diagramme Figma quels sont les trois actifs matériels de l'entreprise qui sont l'objet d'un incident ?
    \item Définir ce qu'est l'actif objet de l'incident numéro 3
    \item Donner sa fonction dans le réseau de l'organisation compte tenu de la panne déclarée
\end{itemize}

\begin{tcolorbox}[colback=cyan!5,colframe=cyan!60!black,title=ITIL]
Selon les bonnes pratiques ITIL, la gestion des incidents suit plusieurs étapes 
\end{tcolorbox}

\begin{itemize}
    \item Quelles sont les étapes que Francis suit (voir le diagramme Figma)
    \item Qu'est-ce que ITIL ?
    \item En observant la définition d'ITIL vous risquez de tomber sur les notions de helpdesk et SLA
    \item Votre collègue qui utilise GLPI fait-il partie du helpdesk de son entreprise ? Justifiez votre réponse.
    \item Qu'est-ce qu'un SLA ?
\end{itemize}





\end{document}