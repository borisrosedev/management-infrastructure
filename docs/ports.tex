\documentclass[a4paper,11pt]{article}
\usepackage[margin=1.8cm]{geometry}

% Encodage & polices
\usepackage[utf8]{inputenc}
\usepackage[T1]{fontenc}
\usepackage{lmodern}

% Langue
\usepackage[french]{babel}

% Couleurs & tableaux
\usepackage[table]{xcolor} % pour \rowcolors
\usepackage{colortbl}
\usepackage{longtable}     % <-- pour l'environnement longtable

% Boîtes et graphiques
\usepackage[most]{tcolorbox}


% Tableaux
\usepackage{array}
\usepackage{tabularx}
\usepackage{booktabs}
\usepackage{pifont}



\begin{document}

\section*{Ports réseaux essentiels pour BTS SIO 1\textsuperscript{re} année}

\begin{tcolorbox}[colback=blue!5,colframe=blue!75!black,title=Introduction]
Les ports réseaux correspondent à des points d’entrée/sortie utilisés par les applications pour communiquer à travers le réseau.  
Voici une liste des principaux ports que tout étudiant en BTS SIO doit connaître en première année.
\end{tcolorbox}

\begin{longtable}{|p{2cm}|p{4cm}|p{8cm}|}
\hline
\textbf{Port} & \textbf{Service / Protocole} & \textbf{Rôle principal} \\
\hline
20, 21 & FTP (File Transfer Protocol) & Transfert de fichiers en clair (non chiffré). \\
\hline
22 & SSH (Secure Shell) & Connexion distante sécurisée à un serveur (remplace Telnet). \\
\hline
23 & Telnet & Connexion distante non sécurisée (ancien protocole, remplacé par SSH). \\
\hline
25 & SMTP (Simple Mail Transfer Protocol) & Envoi de courriels (serveurs de messagerie). \\
\hline
53 & DNS (Domain Name System) & Résolution des noms de domaine en adresses IP. \\
\hline
67, 68 & DHCP (Dynamic Host Configuration Protocol) & Attribution automatique d’adresse IP et paramètres réseau. \\
\hline
80 & HTTP (HyperText Transfer Protocol) & Navigation web (non sécurisé, pas de chiffrement). \\
\hline
110 & POP3 (Post Office Protocol v3) & Récupération des courriels depuis un serveur. \\
\hline
143 & IMAP (Internet Message Access Protocol) & Consultation des courriels (plus moderne que POP3). \\
\hline
161, 162 & SNMP (Simple Network Management Protocol) & Supervision et gestion des équipements réseau. \\
\hline
389 & LDAP (Lightweight Directory Access Protocol) & Annuaire réseau (Active Directory, authentification centralisée). \\
\hline
443 & HTTPS (HTTP Secure) & Navigation web sécurisée (chiffrement TLS/SSL). \\
\hline
445 & SMB (Server Message Block) & Partage de fichiers et d’imprimantes sous Windows. \\
\hline
3389 & RDP (Remote Desktop Protocol) & Bureau à distance Windows. \\
\hline
\end{longtable}



\end{document}


