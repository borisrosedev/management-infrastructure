\documentclass[12pt,a4paper]{report}

% Marges
\usepackage[a4paper,margin=1.5cm]{geometry}

% Couleurs et encadrés
\usepackage[most]{tcolorbox}
\tcbset{colback=gray!5!white,colframe=black!75!black,arc=3mm,boxrule=0.8pt}

% Encodage et langue
\usepackage[utf8]{inputenc}
\usepackage[T1]{fontenc}
\usepackage[french]{babel}

% Mathématiques
\usepackage{amsmath,amssymb}

\begin{document}

\chapter{Notions mathématiques pour la gestion budgétaire du parc informatique}

\begin{tcolorbox}[title={Objectif du chapitre}]
Ce cours présente les \textbf{outils mathématiques de base} nécessaires pour comprendre et appliquer la gestion budgétaire dans un contexte informatique :  
calculs de coûts, amortissement, pourcentages, proportionnalité, analyse de rentabilité et retour sur investissement.
\end{tcolorbox}

\section*{Pourcentages et proportionnalité}

\begin{tcolorbox}[title={Définition}]
Un \textbf{pourcentage} exprime une proportion d’une quantité par rapport à un total, sur une base de 100.
\[
\text{Pourcentage} = \frac{\text{partie}}{\text{total}} \times 100
\]
\end{tcolorbox}

\textbf{Exemple} :  
Un budget IT de 80 000 € inclut 20 000 € pour le matériel.  
\[
\frac{20\,000}{80\,000} \times 100 = 25\% \quad \Rightarrow \quad 25\% \hspace{0.5cm}  \text{du budget est consacré au matériel}Ò.
\]

\section*{Répartition budgétaire}

\begin{tcolorbox}[title={Définition}]
La \textbf{répartition budgétaire} permet de diviser un budget total en différentes catégories selon des pourcentages prédéfinis.
\end{tcolorbox}

\textbf{Exemple} :  
Un budget de 100 000 € réparti ainsi :
\begin{itemize}
  \item 40\% matériel $\Rightarrow 40\,000$ €,
  \item 30\% logiciels $\Rightarrow 30\,000$ €,
  \item 20\% maintenance $\Rightarrow 20\,000$ €,
  \item 10\% innovation $\Rightarrow 10\,000$ €.
\end{itemize}

\section*{Coût total de possession (TCO)}

\begin{tcolorbox}[title={Définition}]
Le \textbf{TCO (Total Cost of Ownership)} correspond à la somme des coûts directs et indirects liés à un actif informatique durant toute sa durée de vie.
\[
TCO = \text{Coût d’acquisition} + \text{Coût de maintenance} + \text{Coût de support} + \text{Coût d’énergie}
\]
\end{tcolorbox}

\textbf{Exemple} :  
Un serveur coûte 10 000 € à l’achat, 2 000 €/an de maintenance, 1 000 €/an d’énergie et dure 5 ans :
\[
TCO = 10\,000 + (2\,000 + 1\,000) \times 5 = 25\,000 \,€
\]

\section*{Amortissement linéaire}

\begin{tcolorbox}[title={Définition}]
L’\textbf{amortissement linéaire} répartit le coût d’un bien de manière égale sur sa durée d’utilisation.
\[
\text{Amortissement annuel} = \frac{\text{Valeur d’achat}}{\text{Durée de vie}}
\]
\end{tcolorbox}

\textbf{Exemple} :  
Un ordinateur de 1 200 € amorti sur 4 ans :
\[
\frac{1\,200}{4} = 300 \,€ \quad \text{par an.}
\]

\section*{Retour sur investissement (ROI)}

\begin{tcolorbox}[title={Définition}]
Le \textbf{ROI (Return On Investment)} mesure la rentabilité d’un projet :
\[
ROI = \frac{\text{Gain net généré par le projet}}{\text{Coût du projet}} \times 100
\]
\end{tcolorbox}

\textbf{Exemple} :  
Un nouvel outil de supervision coûte 20 000 € mais permet d’économiser 30 000 € en pannes évitées :
\[
ROI = \frac{30\,000 - 20\,000}{20\,000} \times 100 = 50\%
\]

\section*{Analyse comparative : CapEx vs OpEx}

\begin{tcolorbox}[title={Définitions}]
\begin{itemize}
  \item \textbf{CapEx (Capital Expenditure)} : investissement à long terme (ex. achat d’un serveur).
  \item \textbf{OpEx (Operational Expenditure)} : dépense de fonctionnement (ex. abonnement cloud mensuel).
\end{itemize}
\end{tcolorbox}

\textbf{Exemple comparatif} :  
\begin{itemize}
  \item Achat d’un serveur : 15 000 € sur 5 ans.
  \item Location cloud : 400 €/mois sur 5 ans = $400 \times 60 = 24\,000$ €.
\end{itemize}

\textbf{Analyse} : CapEx plus économique, mais OpEx plus flexible.


\chapter{Exercices de mathématiques appliqués à la gestion budgétaire du parc informatique}

\begin{tcolorbox}[title={Objectif}]
Ces exercices permettent de mettre en pratique les notions vues dans le chapitre précédent : pourcentages, répartition budgétaire, TCO, amortissement et ROI.  
Chaque exercice est suivi d’une correction détaillée.
\end{tcolorbox}

\section*{Exercice 1 : Pourcentages et répartition}
Un budget informatique annuel est de 120 000 €.  
Il est réparti ainsi :
\begin{itemize}
  \item 35\% pour le matériel,
  \item 25\% pour les logiciels,
  \item 30\% pour la maintenance,
  \item 10\% pour l’innovation.
\end{itemize}

\textbf{Question} : Calculer les montants attribués à chaque poste.

\begin{tcolorbox}[title={Correction}]
\[
\text{Matériel} = 120\,000 \times 0,35 = 42\,000 \,€
\]
\[
\text{Logiciels} = 120\,000 \times 0,25 = 30\,000 \,€
\]
\[
\text{Maintenance} = 120\,000 \times 0,30 = 36\,000 \,€
\]
\[
\text{Innovation} = 120\,000 \times 0,10 = 12\,000 \,€
\]
\end{tcolorbox}

\section*{Exercice 2 : Coût total de possession (TCO)}
Un serveur coûte 8 000 € à l’achat.  
Il génère 1 500 €/an de maintenance et 500 €/an d’électricité.  
Sa durée de vie est de 4 ans.

\textbf{Question} : Calculer son TCO.

\begin{tcolorbox}[title={Correction}]
\[
TCO = \text{Coût d’achat} + (\text{Maintenance annuelle} + \text{Énergie annuelle}) \times \text{Durée}
\]
\[
TCO = 8\,000 + (1\,500 + 500) \times 4
\]
\[
TCO = 8\,000 + 8\,000 = 16\,000 \,€
\]
\end{tcolorbox}

\section*{Exercice 3 : Amortissement linéaire}
Un parc de 50 ordinateurs est acheté pour un coût total de 60 000 €.  
La durée de vie prévue est de 5 ans.

\textbf{Question} : Calculer l’amortissement annuel par ordinateur.

\begin{tcolorbox}[title={Correction}]
\[
\text{Amortissement annuel (parc)} = \frac{60\,000}{5} = 12\,000 \,€ 
\]
\[
\text{Amortissement annuel (par PC)} = \frac{12\,000}{50} = 240 \,€
\]
Chaque ordinateur est amorti à hauteur de 240 € par an.
\end{tcolorbox}

\section*{Exercice 4 : Retour sur investissement (ROI)}
Une entreprise investit 25 000 € dans un outil de supervision.  
Cet outil permet d’économiser 40 000 € par an en temps de panne évité.  

\textbf{Question} : Calculer le ROI au bout d’un an.

\begin{tcolorbox}[title={Correction}]
\[
ROI = \frac{\text{Gain net}}{\text{Coût}} \times 100
\]
\[
ROI = \frac{40\,000 - 25\,000}{25\,000} \times 100 = \frac{15\,000}{25\,000} \times 100 = 60\%
\]
Le ROI est de 60\% la première année.
\end{tcolorbox}

\section*{Exercice 5 : CapEx vs OpEx}
Deux options sont proposées pour une infrastructure informatique :
\begin{itemize}
  \item \textbf{Option A (CapEx)} : achat d’un serveur pour 20 000 €, durée de vie 5 ans.
  \item \textbf{Option B (OpEx)} : location cloud à 500 €/mois pendant 5 ans.
\end{itemize}

\textbf{Question} : Comparer les coûts totaux et déterminer la solution la plus économique.

\begin{tcolorbox}[title={Correction}]
\[
\text{Option A} = 20\,000 \,€
\]
\[
\text{Option B} = 500 \times 12 \times 5 = 30\,000 \,€
\]
\textbf{Analyse} :  
CapEx (20 000 €) est moins coûteux qu’OpEx (30 000 €) sur 5 ans, mais OpEx offre plus de flexibilité.
\end{tcolorbox}


\end{document}
