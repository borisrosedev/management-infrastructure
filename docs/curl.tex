\documentclass[a4paper,12pt]{article}
\usepackage{tcolorbox}
\usepackage{listings}
\usepackage{xcolor}
\usepackage[margin=1.5cm]{geometry}

\lstset{
  basicstyle=\ttfamily\small,
  backgroundcolor=\color{black!5},
  frame=single,
  breaklines=true
}

\begin{document}

\section*{Cours sur \texttt{curl}}

\begin{tcolorbox}[colback=blue!5,colframe=blue!75!black,title={Introduction}]
\texttt{curl} (Client URL) est un outil en ligne de commande qui permet de transférer des données depuis ou vers un serveur en utilisant des protocoles comme HTTP, HTTPS, FTP, SCP, etc.
\end{tcolorbox}

\begin{tcolorbox}[colback=blue!5,colframe=blue!75!black,title={Commandes de base}]
\begin{itemize}
  \item Télécharger une page web :
  \begin{lstlisting}
  curl https://example.com
  \end{lstlisting}

  \item Télécharger un fichier (nom d'origine) :
  \begin{lstlisting}
  curl -O https://example.com/fichier.zip
  \end{lstlisting}

  \item Sauvegarder dans un fichier spécifique :
  \begin{lstlisting}
  curl -o monfichier.html https://example.com
  \end{lstlisting}

  \item Suivre les redirections HTTP :
  \begin{lstlisting}
  curl -L https://example.com
  \end{lstlisting}

  \item Envoyer des données en POST :
  \begin{lstlisting}
  curl -X POST -d "user=admin&pass=1234" https://example.com/login
  \end{lstlisting}

  \item Ajouter un en-tête HTTP :
  \begin{lstlisting}
  curl -H "Authorization: Bearer TOKEN" https://api.example.com/data
  \end{lstlisting}
\end{itemize}
\end{tcolorbox}

\end{document}
