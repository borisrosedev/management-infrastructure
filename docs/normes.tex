
\documentclass[a4paper,11pt]{article}
\usepackage[margin=1.8cm]{geometry}

% Encodage & polices
\usepackage[utf8]{inputenc}
\usepackage[T1]{fontenc}
\usepackage{lmodern}

% Langue
\usepackage[french]{babel}

% Couleurs & tableaux
\usepackage[table]{xcolor} % pour \rowcolors
\usepackage{colortbl}
\usepackage{longtable}     % <-- pour l'environnement longtable

% Boîtes et graphiques
\usepackage[most]{tcolorbox}


% Tableaux
\usepackage{array}
\usepackage{tabularx}
\usepackage{booktabs}
\usepackage{pifont}


\begin{document}



\begin{tcolorbox}[colback=cyan!5,colframe=cyan!60!black,title=Qu'est-ce que l'ISO/IEC 27001 ?]

\textbf{ISO/IEC 27001} est une norme internationale de référence pour la 
\textbf{gestion de la sécurité de l’information}.  
Elle définit un \textit{Système de Management de la Sécurité de l’Information (SMSI)} permettant aux 
organisations de protéger efficacement leurs données sensibles contre les risques de perte, vol, altération ou accès non autorisé.  

\medskip
\textbf{Objectifs principaux :}
\begin{itemize}
  \item Assurer la \textbf{confidentialité}, l’\textbf{intégrité} et la \textbf{disponibilité} de l’information.
  \item Identifier et évaluer les risques liés à la sécurité de l’information.
  \item Mettre en place des contrôles adaptés pour réduire ces risques.
  \item Instaurer une démarche d’\textbf{amélioration continue}.
\end{itemize}

\medskip
\textbf{Éléments clés de la norme :}
\begin{itemize}
  \item \textbf{Contexte organisationnel :} analyse des besoins, des parties prenantes et du périmètre du SMSI.
  \item \textbf{Leadership :} implication de la direction et définition des responsabilités.
  \item \textbf{Planification :} gestion des risques et opportunités.
  \item \textbf{Support :} ressources, compétences, sensibilisation et communication.
  \item \textbf{Exploitation :} mise en œuvre des mesures de sécurité.
  \item \textbf{Évaluation :} audits internes, suivi des performances, revue de direction.
  \item \textbf{Amélioration continue :} actions correctives et évolutives.
\end{itemize}

\medskip
\textbf{Exemple d’application en entreprise :}
\begin{itemize}
  \item Mise en place d’une politique de sécurité informatique.
  \item Contrôle des accès aux systèmes critiques.
  \item Gestion des incidents de sécurité et plan de continuité d’activité.
\end{itemize}

\medskip
\textbf{Avantage :} l’obtention de la certification ISO/IEC 27001 démontre la maturité 
et la fiabilité d’une organisation en matière de cybersécurité, renforçant la confiance des clients et partenaires.
\end{tcolorbox}


\begin{tcolorbox}[colback=teal!5,colframe=teal!60!black,title=Qu'est-ce que l'ISO/IEC 27005 ?]

\textbf{ISO/IEC 27005} est une norme internationale qui fournit des lignes directrices 
pour la \textbf{gestion des risques liés à la sécurité de l’information}.  
Elle est conçue pour accompagner la mise en œuvre d’un \textbf{SMSI (Système de Management de la Sécurité de l’Information)} 
conformément à la norme \textbf{ISO/IEC 27001}.  

\medskip
\textbf{Objectifs principaux :}
\begin{itemize}
  \item Identifier, analyser et évaluer les risques de sécurité de l’information.
  \item Déterminer les mesures de sécurité adaptées pour réduire les risques.
  \item Fournir un cadre méthodologique pour la prise de décision.
  \item Soutenir l’amélioration continue du SMSI.
\end{itemize}

\medskip
\textbf{Étapes clés de la gestion des risques selon ISO/IEC 27005 :}
\begin{enumerate}
  \item \textbf{Établissement du contexte} : définir le périmètre et les critères d’évaluation des risques.
  \item \textbf{Identification des risques} : recenser les actifs, menaces et vulnérabilités.
  \item \textbf{Analyse des risques} : estimer la vraisemblance et l’impact des incidents.
  \item \textbf{Évaluation des risques} : prioriser les risques selon leur criticité.
  \item \textbf{Traitement des risques} : définir et mettre en œuvre des mesures de sécurité.
  \item \textbf{Acceptation et communication des risques} : décision par la direction.
  \item \textbf{Surveillance et réexamen} : mise à jour régulière en fonction des évolutions.
\end{enumerate}

\medskip
\textbf{Lien avec ISO/IEC 27001 :}
\begin{itemize}
  \item ISO/IEC 27001 définit les exigences pour établir un SMSI.
  \item ISO/IEC 27005 fournit la méthodologie pratique pour gérer les risques sur lesquels repose le SMSI.
\end{itemize}

\medskip
\textbf{Exemple concret :}  
Dans une entreprise, ISO/IEC 27005 peut servir à évaluer le risque de fuite de données via une clé USB non chiffrée, 
puis recommander comme traitement l’\textit{interdiction technique des ports USB} ou le \textit{chiffrement obligatoire}.
\end{tcolorbox}


\begin{tcolorbox}[colback=yellow!5,colframe=yellow!60!black,title=Qu'est-ce que l'ITIL ?]

\textbf{ITIL (Information Technology Infrastructure Library)} est un \textit{ensemble de bonnes pratiques} 
pour la \textbf{gestion des services informatiques (ITSM – IT Service Management)}.  
Il fournit un cadre méthodologique permettant aux organisations de concevoir, délivrer, exploiter et améliorer 
leurs services informatiques en alignement avec les besoins métiers.  

\medskip
\textbf{Objectifs principaux :}
\begin{itemize}
  \item Améliorer la \textbf{qualité des services informatiques}.
  \item Optimiser les \textbf{coûts et ressources} liés aux services IT.
  \item Assurer une meilleure \textbf{satisfaction des utilisateurs et clients}.
  \item Instaurer une démarche d’\textbf{amélioration continue}.
\end{itemize}

\medskip
\textbf{Composantes clés d’ITIL (version 4) :}
\begin{itemize}
  \item \textbf{Principes directeurs} : orientation valeur, collaboration, simplicité, optimisation et automatisation.
  \item \textbf{Système de valeur des services (SVS)} : vision globale intégrant gouvernance, pratiques et amélioration continue.
  \item \textbf{Pratiques ITIL} (anciennement processus) : gestion des incidents, gestion des changements, gestion des niveaux de service, gestion des actifs, etc.
  \item \textbf{Chaîne de valeur des services} : activités qui transforment la demande en valeur pour le client.
\end{itemize}

\medskip
\textbf{Exemple concret en entreprise :}
\begin{itemize}
  \item Mise en place d’un \textit{helpdesk} centralisé pour gérer les incidents utilisateurs.
  \item Processus structuré de gestion des changements afin de réduire les risques lors des mises en production.
  \item Suivi des SLA (Service Level Agreements) pour mesurer la qualité des services fournis.
\end{itemize}

\medskip
\textbf{Avantage :} ITIL n’est pas une norme mais un \textit{référentiel de bonnes pratiques} : 
chaque organisation peut l’adapter à son contexte pour structurer son service informatique et créer de la valeur.
\end{tcolorbox}




\end{document}
