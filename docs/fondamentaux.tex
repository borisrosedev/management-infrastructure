
\documentclass[a4paper,11pt]{article}
\usepackage[margin=1.8cm]{geometry}

% Encodage & polices
\usepackage[utf8]{inputenc}
\usepackage[T1]{fontenc}
\usepackage{lmodern}

% Langue
\usepackage[french]{babel}

% Couleurs & tableaux
\usepackage[table]{xcolor} % pour \rowcolors
\usepackage{colortbl}
\usepackage{longtable}     % <-- pour l'environnement longtable

% Boîtes et graphiques
\usepackage[most]{tcolorbox}


% Tableaux
\usepackage{array}
\usepackage{tabularx}
\usepackage{booktabs}
\usepackage{pifont}


\begin{document}

\begin{tcolorbox}[colback=blue!5,colframe=blue!60!black,title=Qu'est-ce qu'un système d'exploitation ?]

Un \textbf{système d’exploitation (OS – Operating System)} est le \textit{logiciel fondamental} qui fait le lien entre 
le matériel d’un ordinateur (processeur, mémoire, disque, périphériques) et les applications utilisées par les utilisateurs.  
Sans système d’exploitation, un ordinateur ne pourrait pas être exploité de manière pratique.  

\medskip
\textbf{Rôles principaux :}
\begin{itemize}
  \item \textbf{Gestion du matériel :} contrôle processeur, mémoire, périphériques via des pilotes.
  \item \textbf{Gestion des processus :} lance, planifie et arrête les programmes.
  \item \textbf{Gestion de la mémoire :} allocation, libération et protection des zones mémoire.
  \item \textbf{Gestion des fichiers :} création, lecture, écriture, suppression et sécurité des fichiers.
  \item \textbf{Interface utilisateur :} interface graphique (GUI) ou ligne de commande (CLI).
  \item \textbf{Sécurité et administration :} comptes utilisateurs, permissions, protection contre les accès non autorisés.
\end{itemize}

\medskip
\textbf{Exemples :} Windows, macOS, Linux, Android, iOS, systèmes embarqués.
\end{tcolorbox}


\begin{tcolorbox}[colback=green!5,colframe=green!60!black,title=Qu'est-ce qu'un périphérique ?]

Un \textbf{périphérique} est un \textit{équipement matériel} connecté à un ordinateur ou à un réseau, 
qui permet d’étendre ses fonctionnalités, d’interagir avec lui, ou de fournir des services aux utilisateurs.  
Dans la gestion du parc informatique, un périphérique est considéré comme une ressource matérielle au même titre 
qu’un poste de travail ou un serveur, et doit être inventorié, suivi et maintenu.  

\medskip
\textbf{Catégories principales :}
\begin{itemize}
  \item \textbf{Périphériques d’entrée :} clavier, souris, scanner, lecteur de carte.
  \item \textbf{Périphériques de sortie :} écran, imprimante, vidéoprojecteur, casque audio.
  \item \textbf{Périphériques de stockage :} clé USB, disque externe, NAS.
  \item \textbf{Périphériques d’entrée/sortie :} imprimante multifonction, carte réseau, smartphone connecté.
\end{itemize}

\medskip
\textbf{Exemples en entreprise :} imprimantes réseau partagées, scanners de documents, NAS pour le stockage centralisé.
\end{tcolorbox}


\begin{tcolorbox}[colback=orange!5,colframe=orange!60!black,title=Qu'est-ce qu'une licence logicielle ?]

Une \textbf{licence logicielle} est un \textit{contrat légal} qui définit les conditions d’utilisation 
d’un logiciel par un utilisateur ou une organisation.  
Elle précise les droits, limitations et obligations liés à l’installation, l’exécution, la copie, la 
modification ou la distribution du logiciel.  

\medskip
\textbf{Rôles d’une licence logicielle :}
\begin{itemize}
  \item Garantir le respect des droits d’auteur du créateur ou de l’éditeur.
  \item Définir le nombre d’utilisateurs ou d’appareils pouvant utiliser le logiciel.
  \item Préciser les restrictions (ex. : interdiction de revente ou de modification).
  \item Déterminer le modèle économique (achat unique, abonnement, open source, freemium, etc.).
\end{itemize}

\medskip
\textbf{Types courants de licences :}
\begin{itemize}
  \item \textbf{Propriétaire :} l’éditeur garde le contrôle (ex. : Windows, Microsoft Office).
  \item \textbf{Libre / Open Source :} accès au code source, modification et redistribution autorisées 
        (ex. : licences GPL, MIT, Apache).
  \item \textbf{Abonnement (SaaS) :} accès au logiciel via un service en ligne moyennant un paiement récurrent (ex. : Office 365, Adobe Creative Cloud).
\end{itemize}

\medskip
\textbf{Exemple en entreprise :} vérifier la conformité des licences pour éviter le \textit{shadow IT} 
et les risques légaux liés à l’utilisation de logiciels non autorisés.
\end{tcolorbox}





\end{document}