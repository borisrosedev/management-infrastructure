
\documentclass[a4paper,11pt]{article}
\usepackage[margin=1.8cm]{geometry}

% Encodage & polices
\usepackage[utf8]{inputenc}
\usepackage[T1]{fontenc}
\usepackage{lmodern}

% Langue
\usepackage[french]{babel}

% Couleurs & tableaux
\usepackage[table]{xcolor} % pour \rowcolors
\usepackage{colortbl}
\usepackage{longtable}     % <-- pour l'environnement longtable

% Boîtes et graphiques
\usepackage[most]{tcolorbox}


% Tableaux
\usepackage{array}
\usepackage{tabularx}
\usepackage{booktabs}
\usepackage{pifont}


\begin{document}

\begin{tcolorbox}[colback=blue!5,colframe=blue!60!black,title=Qu'est-ce qu'un système d'exploitation ?]

Un \textbf{système d’exploitation (OS – Operating System)} est le \textit{logiciel fondamental} qui fait le lien entre 
le matériel d’un ordinateur (processeur, mémoire, disque, périphériques) et les applications utilisées par les utilisateurs.  
Sans système d’exploitation, un ordinateur ne pourrait pas être exploité de manière pratique.  

\medskip
\textbf{Rôles principaux :}
\begin{itemize}
  \item \textbf{Gestion du matériel :} contrôle processeur, mémoire, périphériques via des pilotes.
  \item \textbf{Gestion des processus :} lance, planifie et arrête les programmes.
  \item \textbf{Gestion de la mémoire :} allocation, libération et protection des zones mémoire.
  \item \textbf{Gestion des fichiers :} création, lecture, écriture, suppression et sécurité des fichiers.
  \item \textbf{Interface utilisateur :} interface graphique (GUI) ou ligne de commande (CLI).
  \item \textbf{Sécurité et administration :} comptes utilisateurs, permissions, protection contre les accès non autorisés.
\end{itemize}

\medskip
\textbf{Exemples :} Windows, macOS, Linux, Android, iOS, systèmes embarqués.
\end{tcolorbox}


\begin{tcolorbox}[colback=green!5,colframe=green!60!black,title=Qu'est-ce qu'un périphérique ?]

Un \textbf{périphérique} est un \textit{équipement matériel} connecté à un ordinateur ou à un réseau, 
qui permet d’étendre ses fonctionnalités, d’interagir avec lui, ou de fournir des services aux utilisateurs.  
Dans la gestion du parc informatique, un périphérique est considéré comme une ressource matérielle au même titre 
qu’un poste de travail ou un serveur, et doit être inventorié, suivi et maintenu.  

\medskip
\textbf{Catégories principales :}
\begin{itemize}
  \item \textbf{Périphériques d’entrée :} clavier, souris, scanner, lecteur de carte.
  \item \textbf{Périphériques de sortie :} écran, imprimante, vidéoprojecteur, casque audio.
  \item \textbf{Périphériques de stockage :} clé USB, disque externe, NAS.
  \item \textbf{Périphériques d’entrée/sortie :} imprimante multifonction, carte réseau, smartphone connecté.
\end{itemize}

\medskip
\textbf{Exemples en entreprise :} imprimantes réseau partagées, scanners de documents, NAS pour le stockage centralisé.
\end{tcolorbox}


\begin{tcolorbox}[colback=orange!5,colframe=orange!60!black,title=Qu'est-ce qu'une licence logicielle ?]

Une \textbf{licence logicielle} est un \textit{contrat légal} qui définit les conditions d’utilisation 
d’un logiciel par un utilisateur ou une organisation.  
Elle précise les droits, limitations et obligations liés à l’installation, l’exécution, la copie, la 
modification ou la distribution du logiciel.  

\medskip
\textbf{Rôles d’une licence logicielle :}
\begin{itemize}
  \item Garantir le respect des droits d’auteur du créateur ou de l’éditeur.
  \item Définir le nombre d’utilisateurs ou d’appareils pouvant utiliser le logiciel.
  \item Préciser les restrictions (ex. : interdiction de revente ou de modification).
  \item Déterminer le modèle économique (achat unique, abonnement, open source, freemium, etc.).
\end{itemize}

\medskip
\textbf{Types courants de licences :}
\begin{itemize}
  \item \textbf{Propriétaire :} l’éditeur garde le contrôle (ex. : Windows, Microsoft Office).
  \item \textbf{Libre / Open Source :} accès au code source, modification et redistribution autorisées 
        (ex. : licences GPL, MIT, Apache).
  \item \textbf{Abonnement (SaaS) :} accès au logiciel via un service en ligne moyennant un paiement récurrent (ex. : Office 365, Adobe Creative Cloud).
\end{itemize}

\medskip
\textbf{Exemple en entreprise :} vérifier la conformité des licences pour éviter le \textit{shadow IT} 
et les risques légaux liés à l’utilisation de logiciels non autorisés.
\end{tcolorbox}


\section*{Définition de l’infogérance}

\begin{tcolorbox}[title={Infogérance}]
L’\textbf{infogérance} désigne la délégation totale ou partielle de la gestion du système d’information (SI) d’une organisation à un prestataire externe spécialisé.  
Ce prestataire, appelé \textbf{fournisseur d’infogérance} ou \textbf{MSP (Managed Service Provider)}, prend en charge certaines tâches informatiques afin que l’entreprise cliente puisse se concentrer sur son cœur de métier.
\end{tcolorbox}

\section*{Types d’infogérance}

\begin{tcolorbox}[title={Différentes formes}]
\begin{itemize}
  \item \textbf{Infogérance globale} : externalisation complète du SI (infrastructure, support, sécurité, applications).
  \item \textbf{Infogérance partielle} : externalisation de certains services spécifiques (ex. : maintenance des serveurs, sauvegardes).
  \item \textbf{Infogérance applicative} : gestion des applications métiers (ERP, CRM, bases de données).
  \item \textbf{Infogérance cloud} : gestion des environnements hébergés sur le cloud (AWS, Azure, GCP).
\end{itemize}
\end{tcolorbox}

\section*{Exemple concret}

\begin{tcolorbox}[title={Exemple}]
Une PME ne possède pas d’équipe IT interne.  
Elle confie à une société d’infogérance :
\begin{itemize}
  \item la maintenance de ses serveurs,
  \item la gestion de ses sauvegardes,
  \item la supervision de son réseau,
  \item le support utilisateur (helpdesk).
\end{itemize}
Ainsi, l’entreprise bénéficie d’un suivi professionnel sans recruter un service IT complet.
\end{tcolorbox}

\section*{Avantages de l’infogérance}

\begin{tcolorbox}[title={Bénéfices}]
\begin{itemize}
  \item \textbf{Réduction des coûts} : pas besoin d’embaucher une équipe IT en interne.
  \item \textbf{Expertise} : accès à des spécialistes qualifiés et disponibles.
  \item \textbf{Flexibilité} : possibilité d’adapter le contrat selon les besoins.
  \item \textbf{Disponibilité accrue} : surveillance 24/7 par le prestataire.
  \item \textbf{Concentration sur le métier} : l’entreprise se focalise sur ses activités principales.
\end{itemize}
\end{tcolorbox}

\section*{Inconvénients et risques}

\begin{tcolorbox}[title={Points de vigilance}]
\begin{itemize}
  \item \textbf{Dépendance au prestataire} : perte de contrôle direct sur le SI.
  \item \textbf{Sécurité des données} : risque lié à la confidentialité si le prestataire n’est pas fiable.
  \item \textbf{Contrats rigides} : certains engagements peuvent être difficiles à adapter.
  \item \textbf{Perte de compétences internes} : affaiblissement du savoir-faire IT en interne.
\end{itemize}
\end{tcolorbox}

\section*{Bonnes pratiques pour l’infogérance}

\begin{tcolorbox}[title={Recommandations}]
\begin{itemize}
  \item Définir un contrat clair (SLA : Service Level Agreement) avec indicateurs de performance (temps de réponse, disponibilité).
  \item Vérifier la conformité réglementaire (RGPD, sécurité des données).
  \item Conserver une équipe IT réduite pour garder une gouvernance interne.
  \item Évaluer régulièrement le prestataire.
\end{itemize}
\end{tcolorbox}


\section*{Définition du SaaS}

\begin{tcolorbox}[title={SaaS}]
Le \textbf{SaaS} (\emph{Software as a Service}, ou \textbf{logiciel en tant que service}) est un modèle de distribution de logiciels dans lequel l’application est hébergée sur les serveurs d’un fournisseur et accessible aux utilisateurs via Internet, généralement par un navigateur web.  
L’entreprise n’a pas besoin d’installer, de maintenir ou de gérer le logiciel localement : tout est géré par le fournisseur.
\end{tcolorbox}

\section*{Caractéristiques principales}

\begin{tcolorbox}[title={Propriétés du SaaS}]
\begin{itemize}
  \item \textbf{Accessibilité} : disponible depuis n’importe quel appareil connecté à Internet.
  \item \textbf{Abonnement} : paiement mensuel ou annuel en fonction du nombre d’utilisateurs.
  \item \textbf{Mises à jour automatiques} : le fournisseur applique les correctifs et nouvelles fonctionnalités.
  \item \textbf{Scalabilité} : possibilité d’ajouter ou de retirer facilement des utilisateurs.
  \item \textbf{Multi-tenant} : plusieurs clients partagent la même infrastructure, mais les données sont isolées.
\end{itemize}
\end{tcolorbox}

\section*{Exemples de SaaS}

\begin{tcolorbox}[title={Applications courantes}]
\begin{itemize}
  \item \textbf{Microsoft 365} (Word, Excel, Outlook en ligne),
  \item \textbf{Google Workspace} (Gmail, Google Drive, Docs),
  \item \textbf{Salesforce} (CRM),
  \item \textbf{Dropbox}, \textbf{OneDrive} (stockage cloud),
  \item \textbf{Slack}, \textbf{Zoom} (collaboration et communication).
\end{itemize}
\end{tcolorbox}

\section*{Avantages du SaaS}

\begin{tcolorbox}[title={Bénéfices}]
\begin{itemize}
  \item \textbf{Réduction des coûts} : pas besoin d’investir dans des serveurs ou une équipe de maintenance lourde.
  \item \textbf{Rapidité de déploiement} : utilisation immédiate après souscription.
  \item \textbf{Accessibilité universelle} : compatible multi-plateformes (PC, tablette, mobile).
  \item \textbf{Sécurité et sauvegardes} : assurées par le fournisseur.
  \item \textbf{Flexibilité} : ajustement du nombre de licences selon les besoins.
\end{itemize}
\end{tcolorbox}

\section*{Inconvénients du SaaS}

\begin{tcolorbox}[title={Limites}]
\begin{itemize}
  \item \textbf{Dépendance à Internet} : accès impossible en cas de coupure réseau.
  \item \textbf{Dépendance au fournisseur} : l’entreprise doit faire confiance à l’éditeur pour la sécurité et la disponibilité.
  \item \textbf{Moins de personnalisation} : fonctionnalités standardisées, parfois limitées.
  \item \textbf{Coûts récurrents} : l’abonnement peut coûter plus cher à long terme qu’un achat unique.
\end{itemize}
\end{tcolorbox}

\section*{Comparaison avec d’autres modèles}

\begin{tcolorbox}[title={IaaS, PaaS et SaaS}]
\begin{itemize}
  \item \textbf{IaaS} (\emph{Infrastructure as a Service}) : mise à disposition d’infrastructures (machines virtuelles, stockage, réseau). Ex. : AWS EC2, Microsoft Azure.
  \item \textbf{PaaS} (\emph{Platform as a Service}) : environnement complet pour développer et déployer des applications. Ex. : Google App Engine, Heroku.
  \item \textbf{SaaS} (\emph{Software as a Service}) : logiciels accessibles directement en ligne. Ex. : Gmail, Salesforce.
\end{itemize}
\end{tcolorbox}

\end{document}




\end{document}