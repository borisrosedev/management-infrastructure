\documentclass[a4paper,12pt]{article}
\usepackage{tcolorbox}
\usepackage{listings}
\usepackage{xcolor}
\usepackage[margin=1.5cm]{geometry}

\lstset{
  basicstyle=\ttfamily\small,
  backgroundcolor=\color{black!5},
  frame=single,
  breaklines=true
}

\begin{document}

\section*{Cours sur \texttt{tar}}

\begin{tcolorbox}[colback=green!5,colframe=green!75!black,title={Introduction}]
\texttt{tar} (Tape Archive) est un utilitaire Unix permettant de créer et manipuler des archives. 
Il est souvent combiné avec \texttt{gzip} ou \texttt{bzip2} pour la compression.
\end{tcolorbox}

\begin{tcolorbox}[colback=green!5,colframe=green!75!black,title={Commandes principales}]
\begin{itemize}
  \item Créer une archive :
  \begin{lstlisting}
  tar -cvf archive.tar dossier/
  \end{lstlisting}

  \item Extraire une archive :
  \begin{lstlisting}
  tar -xvf archive.tar
  \end{lstlisting}

  \item Lister le contenu :
  \begin{lstlisting}
  tar -tvf archive.tar
  \end{lstlisting}

  \item Créer une archive compressée (gzip) :
  \begin{lstlisting}
  tar -czvf archive.tar.gz dossier/
  \end{lstlisting}

  \item Extraire une archive gzip :
  \begin{lstlisting}
  tar -xzvf archive.tar.gz
  \end{lstlisting}

  \item Créer une archive compressée (bzip2) :
  \begin{lstlisting}
  tar -cjvf archive.tar.bz2 dossier/
  \end{lstlisting}
\end{itemize}
\end{tcolorbox}

\end{document}
