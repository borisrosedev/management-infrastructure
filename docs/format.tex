\documentclass[11pt,a4paper]{article}

% --- Encodage & polices ---
\usepackage[T1]{fontenc}
\usepackage[utf8]{inputenc}
\usepackage{lmodern}

% --- Mise en page ---
\usepackage[margin=2.5cm]{geometry}

% --- Couleurs et hyperliens ---
\usepackage{xcolor}
\definecolor{linkcolor}{HTML}{0A66C2}
\usepackage[colorlinks=true,linkcolor=linkcolor,urlcolor=linkcolor,citecolor=linkcolor]{hyperref}

% --- Listings pour le code ---
\usepackage{listings}
\lstdefinelanguage{bash}{
  sensitive=true,
  morecomment=[l]{\#},
  morestring=[b]",
}
\lstset{
  language=bash,
  basicstyle=\ttfamily\small,
  numbers=left,
  numberstyle=\scriptsize\color{gray},
  stepnumber=1,
  numbersep=8pt,
  showstringspaces=false,
  columns=fullflexible,
  keepspaces=true,
  tabsize=2,
  breaklines=true,
  frame=single,
  rulecolor=\color{black!20},
  keywordstyle=\bfseries\color{black},
  commentstyle=\itshape\color{gray!70},
  stringstyle=\color{teal!70!black},
}

% --- Listes compactes ---
\usepackage{enumitem}
\setlist[itemize]{nosep,left=0pt..1.5em}
\setlist[enumerate]{nosep,left=0pt..1.5em}

\title{Tout savoir sur les spécificateurs de format en \texttt{printf} (\%b, \%s, etc.)}
\author{}
\date{}

\begin{document}
\maketitle

\section*{Introduction}
Dans le shell Bash (et en C également), la commande \texttt{printf} utilise des \emph{spécificateurs de format}. Ceux-ci permettent de contrôler la manière dont les valeurs sont interprétées et affichées. Les plus courants sont \texttt{\%s}, \texttt{\%d}, \texttt{\%f}, \texttt{\%x}, \texttt{\%o}, et, dans le cas de Bash, un très utile : \texttt{\%b}. 

Nous allons détailler chacun de ces spécificateurs, avec explications et exemples.

\section*{\%s : Chaîne de caractères}
\begin{itemize}
  \item Affiche la valeur comme une \textbf{chaîne brute}.
  \item Si la chaîne contient des séquences comme \texttt{\textbackslash n} ou \texttt{\textbackslash t}, elles \emph{ne sont pas interprétées}.
\end{itemize}

\textbf{Exemple :}
\begin{lstlisting}
printf "%s\n" "Bonjour\nMonde"
\end{lstlisting}

\textbf{Sortie :}
\begin{verbatim}
Bonjour\nMonde
\end{verbatim}

Ici, les caractères \verb|\n| restent littéraux.

\section*{\%b : Chaîne avec interprétation des échappements}
\begin{itemize}
  \item Identique à \texttt{\%s}, mais interprète les séquences d’échappement du style C.
  \item Séquences courantes :
  \begin{itemize}
    \item \verb|\n| : retour à la ligne
    \item \verb|\t| : tabulation
    \item \verb|\033| ou \verb|\e| : caractère ESC (utile pour les couleurs ANSI)
    \item \verb|\\| : un antislash littéral
  \end{itemize}
\end{itemize}

\textbf{Exemple :}
\begin{lstlisting}
printf "%b\n" "Bonjour\nMonde"
\end{lstlisting}

\textbf{Sortie :}
\begin{verbatim}
Bonjour
Monde
\end{verbatim}

\section*{Comparaison directe}
\begin{lstlisting}
printf "%s\n" "Texte\nAvec\nSauts"
# Affiche : Texte\nAvec\nSauts

printf "%b\n" "Texte\nAvec\nSauts"
# Affiche :
# Texte
# Avec
# Sauts
\end{lstlisting}

\section*{Autres spécificateurs utiles}
\begin{itemize}
  \item \texttt{\%d} : entier décimal
  \item \texttt{\%f} : flottant
  \item \texttt{\%x} : entier en hexadécimal
  \item \texttt{\%o} : entier en octal
  \item \texttt{\%c} : caractère unique (correspondant au code ASCII)
\end{itemize}

\textbf{Exemple numérique :}
\begin{lstlisting}
printf "Décimal : %d, Hexa : %x, Octal : %o\n" 65 65 65
\end{lstlisting}

\textbf{Sortie :}
\begin{verbatim}
Décimal : 65, Hexa : 41, Octal : 101
\end{verbatim}

\section*{Largeur et alignement}
\begin{itemize}
  \item \texttt{\%10s} : chaîne alignée à droite dans un champ de 10 caractères.
  \item \texttt{\%-10s} : chaîne alignée à gauche.
  \item \texttt{\%05d} : entier avec remplissage de zéros sur 5 chiffres.
\end{itemize}

\textbf{Exemple :}
\begin{lstlisting}
printf "|%10s|\n" "abc"
printf "|%-10s|\n" "abc"
printf "|%05d|\n" 42
\end{lstlisting}

\textbf{Sortie :}
\begin{verbatim}
|       abc|
|abc       |
|00042|
\end{verbatim}

\section*{Applications pratiques de \%b}
\begin{itemize}
  \item Affichage coloré (séquences ANSI) :
\begin{lstlisting}
RED=$'\033[1;31m'
NC=$'\033[0m'
printf "%b\n" "${RED}Texte rouge${NC}"
\end{lstlisting}

  \item Multi-lignes intégrées :
\begin{lstlisting}
printf "%b" "Ligne1\nLigne2\n"
\end{lstlisting}

\end{itemize}

\section*{Résumé}
\begin{itemize}
  \item \textbf{\%s} : affiche une chaîne littérale.
  \item \textbf{\%b} : affiche une chaîne en interprétant les séquences d’échappement.
  \item Autres spécificateurs (\%d, \%f, \%x, etc.) permettent un contrôle fin sur la présentation des nombres et chaînes.
  \item On peut combiner les options (largeur, alignement, remplissage).
\end{itemize}

\end{document}
