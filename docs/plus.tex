
\documentclass[a4paper,11pt]{article}
\usepackage[margin=1.8cm]{geometry}

% Encodage & polices
\usepackage[utf8]{inputenc}
\usepackage[T1]{fontenc}
\usepackage{lmodern}

% Langue
\usepackage[french]{babel}

% Couleurs & tableaux
\usepackage[table]{xcolor} % pour \rowcolors
\usepackage{colortbl}
\usepackage{longtable}     % <-- pour l'environnement longtable

% Boîtes et graphiques
\usepackage[most]{tcolorbox}


% Tableaux
\usepackage{array}
\usepackage{tabularx}
\usepackage{booktabs}
\usepackage{pifont}


\begin{document}






\begin{tcolorbox}[colback=purple!5,colframe=purple!60!black,title=Qu'est-ce que le FinOps ?]

\textbf{FinOps} (contraction de \textit{Finance} et \textit{DevOps}) est une pratique de gestion qui vise à 
\textbf{optimiser les coûts liés au Cloud} tout en favorisant la collaboration entre les équipes 
\textit{financières}, \textit{techniques} et \textit{opérationnelles}.  

L’objectif principal est d’assurer un \textbf{équilibre entre performance, innovation et maîtrise budgétaire} 
dans l’utilisation des ressources Cloud.  

\medskip
\textbf{Principes clés du FinOps :}
\begin{itemize}
  \item \textbf{Visibilité des coûts :} suivre et analyser la consommation des ressources Cloud.
  \item \textbf{Responsabilisation :} impliquer les équipes techniques dans la maîtrise des dépenses.
  \item \textbf{Optimisation continue :} adapter les services (dimensionnement, arrêt automatique, réservation).
  \item \textbf{Collaboration :} créer un langage commun entre IT, finance et management.
\end{itemize}

\medskip
\textbf{Exemples de pratiques FinOps :}
\begin{itemize}
  \item Mettre en place des tableaux de bord de suivi des dépenses Cloud.
  \item Identifier les ressources sur-provisionnées ou inutilisées et les supprimer.
  \item Comparer les modèles de facturation (\textit{on-demand}, \textit{reserved instances}, \textit{spot}).
\end{itemize}

\end{tcolorbox}


\begin{tcolorbox}[colback=red!5,colframe=red!60!black,title=Qu'est-ce que le Shadow IT ?]

Le \textbf{Shadow IT} désigne l’ensemble des outils, logiciels, applications ou services numériques 
utilisés dans une organisation \textbf{sans validation ni contrôle préalable de la DSI (Direction des Systèmes d’Information)}.  

Ces solutions peuvent être mises en place par les employés pour répondre à un besoin métier 
immédiat (ex. : stockage Cloud personnel, messagerie non officielle, applications collaboratives).  

\medskip
\textbf{Exemples de Shadow IT :}
\begin{itemize}
  \item Utiliser un compte \textit{Dropbox} ou \textit{Google Drive} personnel pour partager des fichiers professionnels.
  \item Installer un logiciel non autorisé sur son poste de travail.
  \item Employer une messagerie instantanée non validée (WhatsApp, Slack non officiel).
  \item Souscrire à un service SaaS sans l’accord de la DSI.
\end{itemize}

\medskip
\textbf{Risques liés au Shadow IT :}
\begin{itemize}
  \item \textbf{Sécurité :} fuite de données, absence de chiffrement, absence de mises à jour.
  \item \textbf{Conformité :} non-respect des réglementations (RGPD, ISO 27001).
  \item \textbf{Gestion des coûts :} abonnements redondants, dépenses non maîtrisées.
  \item \textbf{Support :} absence de garantie et d’intégration avec le SI officiel.
\end{itemize}

\medskip
\textbf{Mesures de prévention :}
\begin{itemize}
  \item Sensibiliser les utilisateurs aux risques du Shadow IT.
  \item Mettre en place des solutions officielles répondant aux besoins métiers.
  \item Surveiller le réseau et les postes pour détecter les usages non conformes.
  \item Encourager la communication entre utilisateurs et DSI.
\end{itemize}

\end{tcolorbox}


\begin{tcolorbox}[colback=gray!5,colframe=gray!60!black,title=Qu'est-ce qu'un Helpdesk ?]

Un \textbf{helpdesk} (ou centre de support) est un \textit{point de contact centralisé} entre les 
utilisateurs d’un système informatique et l’équipe de support technique.  
Il a pour mission de \textbf{répondre aux incidents}, de \textbf{traiter les demandes d’assistance} 
et de \textbf{faciliter la communication} entre les utilisateurs et la DSI.  

\medskip
\textbf{Rôles principaux :}
\begin{itemize}
  \item Enregistrer et classifier les incidents ou demandes utilisateurs.
  \item Fournir une assistance de premier niveau (ex. : mot de passe oublié, problème d’impression).
  \item Escalader les problèmes complexes aux niveaux supérieurs de support.
  \item Suivre et documenter les tickets jusqu’à leur résolution.
  \item Produire des rapports pour améliorer la qualité des services IT.
\end{itemize}

\medskip
\textbf{Niveaux typiques de support :}
\begin{itemize}
  \item \textbf{Niveau 1 :} support de base, résolution des problèmes courants.
  \item \textbf{Niveau 2 :} experts techniques, résolution plus approfondie.
  \item \textbf{Niveau 3 :} spécialistes ou éditeurs, résolution des problèmes complexes.
\end{itemize}

\medskip
\textbf{Exemple en entreprise :}
\begin{itemize}
  \item Un utilisateur ne peut pas se connecter à son compte → création d’un ticket au helpdesk.
  \item Le helpdesk réinitialise son mot de passe (Niveau 1).
  \item Si le problème persiste, le ticket est transféré à l’équipe systèmes/réseaux (Niveau 2).
\end{itemize}

\medskip
\textbf{Outils de helpdesk courants :} GLPI, OTRS, Freshdesk, Zendesk, Jira Service Management.
\end{tcolorbox}


\begin{tcolorbox}[colback=blue!5,colframe=blue!70!black,title=Qu'est-ce qu'un SLA (Service Level Agreement) ?]

Un \textbf{SLA (Service Level Agreement)} ou \textit{accord de niveau de service} est un 
\textbf{contrat formel entre un fournisseur de services et un client} qui définit précisément 
le \textit{niveau de service attendu}.  
Il sert de référence pour mesurer la qualité des prestations fournies et établir des obligations réciproques.  

\medskip
\textbf{Éléments clés d’un SLA :}
\begin{itemize}
  \item \textbf{Description du service :} portée, fonctionnalités, responsabilités.
  \item \textbf{Indicateurs de performance (KPI) :} disponibilité, temps de réponse, taux d’erreurs.
  \item \textbf{Objectifs de service :} taux de disponibilité garanti (ex. 99,9\%), délai moyen de résolution d’incidents.
  \item \textbf{Obligations du client :} respect des procédures de signalement, fourniture des informations nécessaires.
  \item \textbf{Pénalités et compensations :} remises ou crédits en cas de non-respect des engagements.
\end{itemize}

\medskip
\textbf{Exemples concrets :}
\begin{itemize}
  \item Un fournisseur Cloud garantit une disponibilité de son service à \textbf{99,95\%}.
  \item Un helpdesk s’engage à répondre aux tickets critiques sous \textbf{15 minutes}.
  \item Un opérateur télécom garantit un temps de rétablissement maximal de \textbf{4 heures}.
\end{itemize}

\medskip
\textbf{Rôle dans la gestion informatique :}  
Le SLA est essentiel pour assurer la transparence, éviter les malentendus, et établir une base de confiance entre 
l’entreprise et ses prestataires ou services internes.
\end{tcolorbox}




\end{document}