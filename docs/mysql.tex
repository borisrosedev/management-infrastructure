
\documentclass[a4paper,11pt]{article}
\usepackage[margin=1.8cm]{geometry}

% Encodage & polices
\usepackage[utf8]{inputenc}
\usepackage[T1]{fontenc}
\usepackage{lmodern}

% Langue
\usepackage[french]{babel}
\usepackage{minted}

% Couleurs & tableaux

\usepackage[table]{xcolor} % pour \rowcolors
\usepackage{colortbl}
\usepackage{longtable}     % <-- pour l'environnement longtable

% Boîtes et graphiques
\usepackage[most]{tcolorbox}


% Tableaux
\usepackage{array}
\usepackage{tabularx}
\usepackage{booktabs}
\usepackage{pifont}


\title{Cours MySQL}
\author{Formateur: Boris Rose}
\date{\today}

\begin{document}
\maketitle

\section*{1. Introduction à MySQL}

\begin{tcolorbox}[colback=blue!5!white,colframe=blue!75!black,title=Qu'est-ce que MySQL ?]
MySQL est un \textbf{Système de Gestion de Base de Données Relationnelle} (SGBDR). 
Il permet de stocker, organiser et interroger des données. 

Il utilise le langage \textbf{SQL} (Structured Query Language) pour interagir avec les données.
\end{tcolorbox}

\begin{tcolorbox}[colback=green!5!white,colframe=green!75!black,title=Différences avec d'autres SGBD]
\begin{itemize}
    \item \textbf{MySQL} : rapide, open source, largement utilisé pour le web.
    \item \textbf{MariaDB} : fork de MySQL, 100\% compatible.
    \item \textbf{PostgreSQL} : plus avancé pour les fonctions complexes.
\end{itemize}
\end{tcolorbox}

\begin{tcolorbox}[colback=yellow!5!white,colframe=yellow!75!black,title=Installation]
\begin{itemize}
    \item \textbf{Linux (Debian/Ubuntu)} :
    \begin{minted}[fontsize=\small,frame=single,bgcolor=yellow!5]{bash}
sudo apt update
sudo apt install mysql-server
    \end{minted}

    \item \textbf{macOS (Homebrew)} :
    \begin{minted}[fontsize=\small,frame=single,bgcolor=yellow!5]{bash}
brew install mysql
    \end{minted}

    \item \textbf{Windows} : via l’installeur MySQL Workbench.
\end{itemize}
\end{tcolorbox}

\begin{tcolorbox}[colback=red!5!white,colframe=red!75!black,title=Connexion à MySQL]
Pour se connecter en ligne de commande :
\begin{minted}[fontsize=\small,frame=single,bgcolor=red!5]{bash}
mysql -u root -p
\end{minted}
Puis entrer le mot de passe de l’utilisateur \texttt{root}.
\end{tcolorbox}


\section*{1. Introduction à MySQL}

\begin{tcolorbox}[colback=blue!5!white,colframe=blue!75!black,title=Qu'est-ce que MySQL ?]
MySQL est un \textbf{Système de Gestion de Base de Données Relationnelle} (SGBDR). 
Il utilise le langage \textbf{SQL} (Structured Query Language) pour interagir avec les données :
\begin{itemize}
    \item Stockage structuré dans des tables.
    \item Relations entre tables via des clés.
    \item Gestion multi-utilisateurs.
\end{itemize}
\end{tcolorbox}

\begin{tcolorbox}[colback=yellow!5!white,colframe=yellow!75!black,title=Connexion à MySQL]
\begin{minted}[fontsize=\small,frame=single,bgcolor=yellow!5]{bash}
mysql -u root -p
\end{minted}
\end{tcolorbox}

\section*{2. Création et gestion des bases de données}

\begin{tcolorbox}[colback=green!5!white,colframe=green!75!black,title=Créer une base de données]
\begin{minted}[fontsize=\small,frame=single,bgcolor=green!5]{sql}
CREATE DATABASE entreprise;
\end{minted}
\end{tcolorbox}

\begin{tcolorbox}[colback=green!5!white,colframe=green!75!black,title=Afficher les bases existantes]
\begin{minted}[fontsize=\small,frame=single,bgcolor=green!5]{sql}
SHOW DATABASES;
\end{minted}
\end{tcolorbox}

\begin{tcolorbox}[colback=green!5!white,colframe=green!75!black,title=Utiliser une base de données]
\begin{minted}[fontsize=\small,frame=single,bgcolor=green!5]{sql}
USE entreprise;
\end{minted}
\end{tcolorbox}

\begin{tcolorbox}[colback=red!5!white,colframe=red!75!black,title=Supprimer une base de données]
\begin{minted}[fontsize=\small,frame=single,bgcolor=red!5]{sql}
DROP DATABASE entreprise;
\end{minted}
\end{tcolorbox}


\section*{2. Création de tables et manipulation des données (CRUD)}

\begin{tcolorbox}[colback=blue!5!white,colframe=blue!75!black,title=Créer une table]
\begin{minted}[fontsize=\small,frame=single,bgcolor=blue!5]{sql}
CREATE TABLE employes (
    id INT AUTO_INCREMENT PRIMARY KEY,
    nom VARCHAR(50) NOT NULL,
    prenom VARCHAR(50) NOT NULL,
    poste VARCHAR(50),
    salaire DECIMAL(10,2)
);
\end{minted}
\end{tcolorbox}

\begin{tcolorbox}[colback=green!5!white,colframe=green!75!black,title=Insérer des données (INSERT)]
\begin{minted}[fontsize=\small,frame=single,bgcolor=green!5]{sql}
INSERT INTO employes (nom, prenom, poste, salaire)
VALUES ('Dupont', 'Jean', 'Développeur', 2500.00);

INSERT INTO employes (nom, prenom, poste, salaire)
VALUES ('Martin', 'Sophie', 'Chef de projet', 3200.00);
\end{minted}
\end{tcolorbox}

\begin{tcolorbox}[colback=yellow!5!white,colframe=yellow!75!black,title=Lire les données (SELECT)]
\begin{minted}[fontsize=\small,frame=single,bgcolor=yellow!5]{sql}
-- Tous les enregistrements
SELECT * FROM employes;

-- Avec condition
SELECT nom, prenom, salaire
FROM employes
WHERE salaire > 3000;
\end{minted}
\end{tcolorbox}

\begin{tcolorbox}[colback=orange!5!white,colframe=orange!75!black,title=Modifier les données (UPDATE)]
\begin{minted}[fontsize=\small,frame=single,bgcolor=orange!5]{sql}
UPDATE employes
SET salaire = 2800
WHERE nom = 'Dupont';
\end{minted}
\end{tcolorbox}

\begin{tcolorbox}[colback=red!5!white,colframe=red!75!black,title=Supprimer des données (DELETE)]
\begin{minted}[fontsize=\small,frame=single,bgcolor=red!5]{sql}
DELETE FROM employes
WHERE nom = 'Martin';
\end{minted}
\end{tcolorbox}


\section*{3. Requêtes avancées (jointures, regroupements, sous-requêtes)}

\begin{tcolorbox}[colback=blue!5!white,colframe=blue!75!black,title=Jointures internes (INNER JOIN)]
\begin{minted}[fontsize=\small,frame=single,bgcolor=blue!5]{sql}
-- Exemple avec deux tables : employes et departements
SELECT e.nom, e.prenom, d.nom AS departement
FROM employes e
INNER JOIN departements d ON e.departement_id = d.id;
\end{minted}
\end{tcolorbox}

\begin{tcolorbox}[colback=green!5!white,colframe=green!75!black,title=Jointures externes (LEFT / RIGHT JOIN)]
\begin{minted}[fontsize=\small,frame=single,bgcolor=green!5]{sql}
-- LEFT JOIN : tous les employés, même sans département
SELECT e.nom, e.prenom, d.nom AS departement
FROM employes e
LEFT JOIN departements d ON e.departement_id = d.id;

-- RIGHT JOIN : tous les départements, même sans employés
SELECT e.nom, e.prenom, d.nom AS departement
FROM employes e
RIGHT JOIN departements d ON e.departement_id = d.id;
\end{minted}
\end{tcolorbox}

\begin{tcolorbox}[colback=yellow!5!white,colframe=yellow!75!black,title=Regroupement et agrégats (GROUP BY/ HAVING)]
\begin{minted}[fontsize=\small,frame=single,bgcolor=yellow!5]{sql}
-- Salaire moyen par département
SELECT d.nom AS departement, AVG(e.salaire) AS salaire_moyen
FROM employes e
INNER JOIN departements d ON e.departement_id = d.id
GROUP BY d.nom;

-- Filtrer les départements dont le salaire moyen > 3000
SELECT d.nom AS departement, AVG(e.salaire) AS salaire_moyen
FROM employes e
INNER JOIN departements d ON e.departement_id = d.id
GROUP BY d.nom
HAVING AVG(e.salaire) > 3000;
\end{minted}
\end{tcolorbox}

\begin{tcolorbox}[colback=orange!5!white,colframe=orange!75!black,title=Sous-requêtes (Subqueries)]
\begin{minted}[fontsize=\small,frame=single,bgcolor=orange!5]{sql}
-- Employés dont le salaire est supérieur à la moyenne
SELECT nom, prenom, salaire
FROM employes
WHERE salaire > (SELECT AVG(salaire) FROM employes);

-- Départements ayant plus de 5 employés
SELECT nom
FROM departements
WHERE id IN (
    SELECT departement_id
    FROM employes
    GROUP BY departement_id
    HAVING COUNT(*) > 5
);
\end{minted}
\end{tcolorbox}

\begin{tcolorbox}[colback=red!5!white,colframe=red!75!black,title=Tri et limitation (ORDER BY / LIMIT)]
\begin{minted}[fontsize=\small,frame=single,bgcolor=red!5]{sql}
-- Trier par salaire décroissant
SELECT nom, prenom, salaire
FROM employes
ORDER BY salaire DESC;

-- Obtenir les 3 plus hauts salaires
SELECT nom, prenom, salaire
FROM employes
ORDER BY salaire DESC
LIMIT 3;
\end{minted}
\end{tcolorbox}


\section*{4. Contraintes, clés étrangères et index}

\begin{tcolorbox}[colback=blue!5!white,colframe=blue!75!black,title=Contraintes de base (PRIMARY KEY | NOT NULL | UNIQUE | DEFAULT)]
\begin{minted}[fontsize=\small,frame=single,bgcolor=blue!5]{sql}
-- Création d'une table avec contraintes
CREATE TABLE departements (
  id INT PRIMARY KEY AUTO_INCREMENT,
  nom VARCHAR(100) NOT NULL UNIQUE,
  actif TINYINT(1) NOT NULL DEFAULT 1
);
\end{minted}
\end{tcolorbox}

\begin{tcolorbox}[colback=green!5!white,colframe=green!75!black,title=Clés étrangères (FOREIGN KEY) avec actions]
\begin{minted}[fontsize=\small,frame=single,bgcolor=green!5]{sql}
-- InnoDB requis pour les FKs
CREATE TABLE employes (
  id INT PRIMARY KEY AUTO_INCREMENT,
  nom VARCHAR(50) NOT NULL,
  prenom VARCHAR(50) NOT NULL,
  salaire DECIMAL(10,2) NOT NULL,
  departement_id INT NULL,
  CONSTRAINT fk_employes_dept
    FOREIGN KEY (departement_id)
    REFERENCES departements(id)
    ON DELETE SET NULL   -- alternatives: RESTRICT | CASCADE
    ON UPDATE CASCADE
) ENGINE=InnoDB;
\end{minted}

\begin{minted}[fontsize=\small,frame=single,bgcolor=green!5]{sql}
-- Ajouter une contrainte après création
ALTER TABLE employes
ADD CONSTRAINT fk_employes_dept
FOREIGN KEY (departement_id) REFERENCES departements(id)
ON DELETE SET NULL ON UPDATE CASCADE;

-- Supprimer une contrainte de clé étrangère
ALTER TABLE employes
DROP FOREIGN KEY fk_employes_dept;
\end{minted}
\end{tcolorbox}

\begin{tcolorbox}[colback=yellow!5!white,colframe=yellow!75!black,title=Clés primaires composées et contraintes composées]
\begin{minted}[fontsize=\small,frame=single,bgcolor=yellow!5]{sql}
-- Exemple de PK composite
CREATE TABLE affectations (
  employe_id INT NOT NULL,
  projet_id  INT NOT NULL,
  role VARCHAR(40) NOT NULL,
  PRIMARY KEY (employe_id, projet_id),         -- composite
  CONSTRAINT fk_aff_emp FOREIGN KEY (employe_id) REFERENCES employes(id)
    ON DELETE CASCADE ON UPDATE CASCADE,
  CONSTRAINT fk_aff_proj FOREIGN KEY (projet_id)  REFERENCES projets(id)
    ON DELETE CASCADE ON UPDATE CASCADE,
  CONSTRAINT uq_role_par_projet UNIQUE (projet_id, role)
) ENGINE=InnoDB;
\end{minted}
\end{tcolorbox}

\begin{tcolorbox}[colback=orange!5!white,colframe=orange!75!black,title=CHECK (MySQL 8.0+)]
\begin{minted}[fontsize=\small,frame=single,bgcolor=orange!5]{sql}
-- Les CHECK sont appliqués depuis MySQL 8.0.16+
ALTER TABLE employes
ADD CONSTRAINT chk_salaire_positif CHECK (salaire >= 0);

-- Exemple multi-condition
ALTER TABLE employes
ADD CONSTRAINT chk_noms CHECK (CHAR_LENGTH(nom) >= 2 AND CHAR_LENGTH(prenom) >= 2);
\end{minted}
\end{tcolorbox}

\begin{tcolorbox}[colback=teal!5!white,colframe=teal!60!black,title=Index: créer/ supprimer /lister]
\begin{minted}[fontsize=\small,frame=single,bgcolor=teal!5]{sql}
-- Index simple (BTREE par défaut)
CREATE INDEX idx_emp_salaire ON employes (salaire);

-- Index composite (l'ordre des colonnes compte pour les recherches)
CREATE INDEX idx_emp_nom_prenom ON employes (nom, prenom);

-- Index préfixe (utile sur VARCHAR longs)
CREATE INDEX idx_emp_nom_prefix ON employes (nom(20));

-- Supprimer un index
DROP INDEX idx_emp_salaire ON employes;

-- Voir les index d'une table
SHOW INDEX FROM employes;
\end{minted}
\end{tcolorbox}

\begin{tcolorbox}[colback=purple!5!white,colframe=purple!70!black,title=Index spécialisés: FULLTEXT et UNIQUE]
\begin{minted}[fontsize=\small,frame=single,bgcolor=purple!5]{sql}
-- Index UNIQUE (empêche les doublons)
CREATE UNIQUE INDEX uq_dept_nom ON departements(nom);

-- FULLTEXT (InnoDB/MyISAM, colonnes TEXT/VARCHAR)
CREATE FULLTEXT INDEX ft_emp_nom_prenom ON employes (nom, prenom);

-- Recherche plein texte (mode naturel)
SELECT id, nom, prenom
FROM employes
WHERE MATCH(nom, prenom) AGAINST ('"Jean Dupont"' IN NATURAL LANGUAGE MODE);

-- Mode booléen (opérateurs +, -, *)
SELECT id, nom, prenom
FROM employes
WHERE MATCH(nom, prenom) AGAINST ('+Jean +dev* -stagiaire' IN BOOLEAN MODE);
\end{minted}
\end{tcolorbox}

\begin{tcolorbox}[colback=red!5!white,colframe=red!75!black,title=Bonnes pratiques express]
\begin{itemize}
  \item Utiliser \texttt{ENGINE=InnoDB} pour le support ACID et les clés étrangères.
  \item Indexer les colonnes utilisées dans les \texttt{JOIN}, \texttt{WHERE}, \texttt{ORDER BY}.
  \item Préférer des index composés alignés avec les prédicats les plus sélectifs en premier.
  \item Éviter de multiplier les index redondants (coût d’écriture, stockage).
  \item Vérifier l’impact avec \texttt{EXPLAIN} (vu dans la partie Performance).
\end{itemize}
\end{tcolorbox}


\section*{5. Transactions et gestion de la concurrence}

\begin{tcolorbox}[colback=blue!5!white,colframe=blue!75!black,title=Notion de transaction]
Une \textbf{transaction} est une suite d’opérations SQL exécutées comme une unité logique.  
Les transactions respectent les propriétés \textbf{ACID} :
\begin{itemize}
  \item \textbf{Atomicité} : tout ou rien.
  \item \textbf{Cohérence} : l’état final doit être valide.
  \item \textbf{Isolation} : les transactions ne se perturbent pas mutuellement.
  \item \textbf{Durabilité} : une fois validée, une transaction survit aux pannes.
\end{itemize}
\end{tcolorbox}

\begin{tcolorbox}[colback=green!5!white,colframe=green!75!black,title=Syntaxe de base des transactions]
\begin{minted}[fontsize=\small,frame=single,bgcolor=green!5]{sql}
START TRANSACTION;

UPDATE comptes SET solde = solde - 100 WHERE id = 1;
UPDATE comptes SET solde = solde + 100 WHERE id = 2;

COMMIT;   -- Valider
-- ou
ROLLBACK; -- Annuler
\end{minted}
\end{tcolorbox}

\begin{tcolorbox}[colback=yellow!5!white,colframe=yellow!75!black,title=Points de sauvegarde (SAVEPOINT)]
\begin{minted}[fontsize=\small,frame=single,bgcolor=yellow!5]{sql}
START TRANSACTION;

UPDATE comptes SET solde = solde - 100 WHERE id = 1;
SAVEPOINT apres_debit;

UPDATE comptes SET solde = solde + 100 WHERE id = 2;

ROLLBACK TO apres_debit; -- Annule le crédit mais conserve le débit
COMMIT;
\end{minted}
\end{tcolorbox}

\begin{tcolorbox}[colback=orange!5!white,colframe=orange!75!black,title=Niveaux d’isolation]
MySQL propose plusieurs niveaux d’isolation pour contrôler la visibilité des changements :
\begin{itemize}
  \item \textbf{READ UNCOMMITTED} : peut lire des données non validées (\textit{dirty reads}).
  \item \textbf{READ COMMITTED} : lit uniquement les données validées.
  \item \textbf{REPEATABLE READ} (par défaut InnoDB) : garantit des lectures cohérentes pendant la transaction.
  \item \textbf{SERIALIZABLE} : le plus strict, exécute les transactions comme si elles étaient séquentielles.
\end{itemize}

\begin{minted}[fontsize=\small,frame=single,bgcolor=orange!5]{sql}
-- Vérifier le niveau actuel
SELECT @@transaction_isolation;

-- Changer le niveau
SET TRANSACTION ISOLATION LEVEL SERIALIZABLE;
\end{minted}
\end{tcolorbox}

\begin{tcolorbox}[colback=red!5!white,colframe=red!75!black,title=Verrous (Locks)]
\begin{itemize}
  \item \textbf{Shared locks (S)} : plusieurs lectures simultanées.
  \item \textbf{Exclusive locks (X)} : écriture exclusive.
  \item \textbf{Row-level locking} : InnoDB verrouille au niveau ligne (meilleure concurrence).
  \item \textbf{Table-level locking} : MyISAM verrouille toute la table.
\end{itemize}

\begin{minted}[fontsize=\small,frame=single,bgcolor=red!5]{sql}
-- Verrouiller une table
LOCK TABLES employes WRITE;

-- Opérations critiques ici

UNLOCK TABLES;
\end{minted}
\end{tcolorbox}

\begin{tcolorbox}[colback=teal!5!white,colframe=teal!60!black,title=Bonnes pratiques transactions]
\begin{itemize}
  \item Garder les transactions \textbf{courtes} pour réduire la contention.
  \item Toujours gérer explicitement \texttt{COMMIT} ou \texttt{ROLLBACK}.
  \item Préférer l’\textbf{isolation par défaut REPEATABLE READ} pour un bon compromis.
  \item Surveiller les verrous bloquants via \texttt{SHOW ENGINE INNODB STATUS}.
\end{itemize}
\end{tcolorbox}


\section*{6. Fonctions, vues et procédures stockées}

\begin{tcolorbox}[colback=blue!5!white,colframe=blue!75!black,title=Fonctions intégrées utiles]
MySQL propose de nombreuses fonctions pour manipuler les données :
\begin{itemize}
  \item \textbf{Chaînes} : \texttt{CONCAT()}, \texttt{UPPER()}, \texttt{LOWER()}, \texttt{SUBSTRING()}.
  \item \textbf{Numériques} : \texttt{ROUND()}, \texttt{CEIL()}, \texttt{FLOOR()}, \texttt{MOD()}.
  \item \textbf{Dates} : \texttt{NOW()}, \texttt{CURDATE()}, \texttt{DATE\_ADD()}, \texttt{DATEDIFF()}.
  \item \textbf{Agrégats} : \texttt{COUNT()}, \texttt{SUM()}, \texttt{AVG()}, \texttt{MIN()}, \texttt{MAX()}.
\end{itemize}

\begin{minted}[fontsize=\small,frame=single,bgcolor=blue!5]{sql}
-- Exemple d'utilisation
SELECT nom, UPPER(prenom), ROUND(salaire, 0)
FROM employes
WHERE YEAR(CURDATE()) - YEAR(date_embauche) > 5;
\end{minted}
\end{tcolorbox}

\begin{tcolorbox}[colback=green!5!white,colframe=green!75!black,title=Créer une vue]
Une \textbf{vue} est une requête enregistrée comme une table virtuelle.

\begin{minted}[fontsize=\small,frame=single,bgcolor=green!5]{sql}
CREATE VIEW vue_employes_high_salary AS
SELECT nom, prenom, salaire
FROM employes
WHERE salaire > 3000;

-- Utilisation
SELECT * FROM vue_employes_high_salary;
\end{minted}

\textbf{Note} : Les vues peuvent être utilisées pour simplifier les requêtes ou restreindre l’accès aux colonnes.
\end{tcolorbox}

\begin{tcolorbox}[colback=yellow!5!white,colframe=yellow!75!black,title=Procédures stockées]
Une \textbf{procédure stockée} est un bloc SQL enregistré dans le serveur.

\begin{minted}[fontsize=\small,frame=single,bgcolor=yellow!5]{sql}
DELIMITER $$

CREATE PROCEDURE augmenter_salaire(IN emp_id INT, IN pourcentage DECIMAL(5,2))
BEGIN
  UPDATE employes
  SET salaire = salaire * (1 + pourcentage/100)
  WHERE id = emp_id;
END $$

DELIMITER ;

-- Appel de la procédure
CALL augmenter_salaire(1, 10);
\end{minted}
\end{tcolorbox}

\begin{tcolorbox}[colback=orange!5!white,colframe=orange!75!black,title=Fonctions stockées]
Une \textbf{fonction stockée} retourne une valeur unique.

\begin{minted}[fontsize=\small,frame=single,bgcolor=orange!5]{sql}
DELIMITER $$

CREATE FUNCTION salaire_annuel(emp_id INT) RETURNS DECIMAL(10,2)
DETERMINISTIC
BEGIN
  DECLARE annuel DECIMAL(10,2);
  SELECT salaire * 12 INTO annuel
  FROM employes
  WHERE id = emp_id;
  RETURN annuel;
END $$

DELIMITER ;

-- Utilisation
SELECT nom, prenom, salaire_annuel(id) AS revenu
FROM employes;
\end{minted}
\end{tcolorbox}

\begin{tcolorbox}[colback=red!5!white,colframe=red!75!black,title=Déclencheurs (TRIGGERS)]
Les \textbf{triggers} permettent d’exécuter automatiquement une action lors d’un événement.

\begin{minted}[fontsize=\small,frame=single,bgcolor=red!5]{sql}
DELIMITER $$

CREATE TRIGGER avant_insert_employe
BEFORE INSERT ON employes
FOR EACH ROW
BEGIN
  IF NEW.salaire < 0 THEN
    SET NEW.salaire = 0;
  END IF;
END $$

DELIMITER ;
\end{minted}

\textbf{Exemple} : Ici, on empêche l’insertion d’un salaire négatif.
\end{tcolorbox}

\section*{7. Sécurité et gestion des utilisateurs}

\begin{tcolorbox}[colback=blue!5!white,colframe=blue!75!black,title=Création et gestion des utilisateurs]
En MySQL, chaque utilisateur est défini par un couple \texttt{utilisateur@hôte}.  
\begin{minted}[fontsize=\small,frame=single,bgcolor=blue!5]{sql}
-- Créer un utilisateur
CREATE USER 'alice'@'localhost' IDENTIFIED BY 'motdepasse';

-- Créer un utilisateur accessible depuis n'importe quelle machine
CREATE USER 'bob'@'%' IDENTIFIED BY 'motdepasse';
\end{minted}
\end{tcolorbox}

\begin{tcolorbox}[colback=green!5!white,colframe=green!75!black,title=Accorder des privilèges (GRANT)]
\begin{minted}[fontsize=\small,frame=single,bgcolor=green!5]{sql}
-- Donner tous les droits sur une base
GRANT ALL PRIVILEGES ON entreprise.* TO 'alice'@'localhost';

-- Donner uniquement SELECT et INSERT sur une table
GRANT SELECT, INSERT ON entreprise.employes TO 'bob'@'%';

-- Appliquer les changements
FLUSH PRIVILEGES;
\end{minted}
\end{tcolorbox}

\begin{tcolorbox}[colback=yellow!5!white,colframe=yellow!75!black,title=Révoquer des privilèges (REVOKE)]
\begin{minted}[fontsize=\small,frame=single,bgcolor=yellow!5]{sql}
-- Retirer le droit INSERT à Bob
REVOKE INSERT ON entreprise.employes FROM 'bob'@'%';
\end{minted}
\end{tcolorbox}

\begin{tcolorbox}[colback=orange!5!white,colframe=orange!75!black,title=Gestion des mots de passe]
\begin{minted}[fontsize=\small,frame=single,bgcolor=orange!5]{sql}
-- Changer le mot de passe
ALTER USER 'alice'@'localhost' IDENTIFIED BY 'nouveauMDP';

-- Forcer la réinitialisation du mot de passe à la prochaine connexion
ALTER USER 'bob'@'%' PASSWORD EXPIRE;
\end{minted}
\end{tcolorbox}

\begin{tcolorbox}[colback=purple!5!white,colframe=purple!75!black,title=Rôles (MySQL 8.0+)]
\begin{minted}[fontsize=\small,frame=single,bgcolor=purple!5]{sql}
-- Créer un rôle
CREATE ROLE lecteur;

-- Accorder des droits au rôle
GRANT SELECT ON entreprise.* TO lecteur;

-- Attribuer un rôle à un utilisateur
GRANT lecteur TO 'bob'@'%';

-- Activer un rôle par défaut
SET DEFAULT ROLE lecteur TO 'bob'@'%';
\end{minted}
\end{tcolorbox}

\begin{tcolorbox}[colback=red!5!white,colframe=red!75!black,title=Bonnes pratiques sécurité]
\begin{itemize}
  \item Ne jamais utiliser \texttt{root} pour les applications.
  \item Créer des utilisateurs spécifiques avec le minimum de droits nécessaires.
  \item Utiliser des mots de passe forts et, si possible, l’authentification par plugin (\texttt{caching\_sha2\_password}).
  \item Surveiller les connexions via les logs MySQL.
\end{itemize}
\end{tcolorbox}


\section*{8. Sauvegarde, restauration et optimisation}

\begin{tcolorbox}[colback=blue!5!white,colframe=blue!75!black,title=Sauvegarde avec mysqldump]
L’outil \texttt{mysqldump} permet de sauvegarder une base ou l’ensemble des bases.
\begin{minted}[fontsize=\small,frame=single,bgcolor=blue!5]{bash}
# Sauvegarder une base
mysqldump -u root -p entreprise > entreprise.sql

# Sauvegarder toutes les bases
mysqldump -u root -p --all-databases > backup_all.sql

# Sauvegarder uniquement la structure sans données
mysqldump -u root -p --no-data entreprise > structure.sql
\end{minted}
\end{tcolorbox}

\begin{tcolorbox}[colback=green!5!white,colframe=green!75!black,title=Restauration]
\begin{minted}[fontsize=\small,frame=single,bgcolor=green!5]{bash}
# Restaurer une base
mysql -u root -p entreprise < entreprise.sql

# Restaurer toutes les bases
mysql -u root -p < backup_all.sql
\end{minted}
\end{tcolorbox}

\begin{tcolorbox}[colback=yellow!5!white,colframe=yellow!75!black,title=Optimisation des requêtes avec EXPLAIN]
La commande \texttt{EXPLAIN} permet de comprendre comment MySQL exécute une requête.
\begin{minted}[fontsize=\small,frame=single,bgcolor=yellow!5]{sql}
EXPLAIN SELECT nom, prenom
FROM employes
WHERE salaire > 3000
ORDER BY nom;
\end{minted}
Elle affiche l’utilisation des index, les scans de table et les jointures.
\end{tcolorbox}

\begin{tcolorbox}[colback=orange!5!white,colframe=orange!75!black,title=Analyse et optimisation des tables]
\begin{minted}[fontsize=\small,frame=single,bgcolor=orange!5]{sql}
-- Vérifier et réparer une table
CHECK TABLE employes;
REPAIR TABLE employes;

-- Optimiser une table (reconstruction, libération d’espace)
OPTIMIZE TABLE employes;
\end{minted}
\end{tcolorbox}

\begin{tcolorbox}[colback=purple!5!white,colframe=purple!75!black,title=Index et performance]
\begin{itemize}
  \item Utiliser des index sur les colonnes de filtrage et jointure.
  \item Éviter les colonnes avec faible sélectivité (ex: booléens).
  \item Préférer les index composés alignés avec les clauses WHERE.
  \item Surveiller les requêtes lentes via \texttt{slow query log}.
\end{itemize}
\end{tcolorbox}

\begin{tcolorbox}[colback=red!5!white,colframe=red!75!black,title=Bonnes pratiques de sauvegarde]
\begin{itemize}
  \item Automatiser les sauvegardes avec des scripts (cron, systemd timer).
  \item Stocker les sauvegardes sur un autre serveur ou dans le cloud.
  \item Tester régulièrement la restauration (sauvegarde inutile si non testée).
  \item Utiliser \texttt{--single-transaction} avec InnoDB pour des dumps cohérents sans bloquer les tables.
\end{itemize}
\end{tcolorbox}


\section*{9. Réplication et haute disponibilité}

\begin{tcolorbox}[colback=blue!5!white,colframe=blue!75!black,title=Principe de la réplication MySQL]
La réplication permet de copier automatiquement les données d’un \textbf{serveur maître} vers un ou plusieurs \textbf{serveurs esclaves} :
\begin{itemize}
  \item \textbf{Master-Slave} : le maître reçoit les écritures, les esclaves répliquent en lecture seule.
  \item \textbf{Master-Master} : chaque serveur peut écrire, mais nécessite une bonne gestion des conflits.
  \item \textbf{Asynchrone} (par défaut) ou \textbf{semi-synchrone}.
\end{itemize}
Avantages : tolérance de panne, montée en charge (lecture sur les esclaves), sauvegarde sans impacter la production.
\end{tcolorbox}

\begin{tcolorbox}[colback=green!5!white,colframe=green!75!black,title=Configuration côté maître]
\begin{minted}[fontsize=\small,frame=single,bgcolor=green!5]{sql}
-- Éditer my.cnf (ou mysqld.cnf) et activer :
[mysqld]
server-id = 1
log_bin = /var/log/mysql/mysql-bin.log

-- Redémarrer MySQL puis créer un utilisateur dédié :
CREATE USER 'repl'@'%' IDENTIFIED BY 'motdepasse';
GRANT REPLICATION SLAVE ON *.* TO 'repl'@'%';
FLUSH PRIVILEGES;

-- Verrouiller les tables pour snapshot cohérent
FLUSH TABLES WITH READ LOCK;

-- Vérifier les infos de binlog
SHOW MASTER STATUS;
\end{minted}
\end{tcolorbox}

\begin{tcolorbox}[colback=yellow!5!white,colframe=yellow!75!black,title=Configuration côté esclave]
\begin{minted}[fontsize=\small,frame=single,bgcolor=yellow!5]{sql}
-- Dans my.cnf
[mysqld]
server-id = 2

-- Lancer la réplication
CHANGE MASTER TO
  MASTER_HOST='ip_master',
  MASTER_USER='repl',
  MASTER_PASSWORD='motdepasse',
  MASTER_LOG_FILE='mysql-bin.000001',
  MASTER_LOG_POS=154;

START SLAVE;

-- Vérifier l'état
SHOW SLAVE STATUS\G
\end{minted}
\end{tcolorbox}

\begin{tcolorbox}[colback=orange!5!white,colframe=orange!75!black,title=Haute disponibilité avec MySQL]
\begin{itemize}
  \item \textbf{Réplication + failover} : basculer vers un esclave en cas de panne du maître.
  \item \textbf{MySQL Router} ou \textbf{ProxySQL} pour rediriger les connexions vers les nœuds disponibles.
  \item \textbf{Group Replication} (MySQL 5.7+) : cluster multi-maîtres avec consensus automatique.
  \item \textbf{InnoDB Cluster} : solution clé en main de haute disponibilité (MySQL Shell + Group Replication + Router).
\end{itemize}
\end{tcolorbox}

\begin{tcolorbox}[colback=red!5!white,colframe=red!75!black,title=Bonnes pratiques réplication et HA]
\begin{itemize}
  \item Sécuriser le compte de réplication avec des privilèges minimaux.
  \item Surveiller la latence de réplication (\texttt{Seconds\_Behind\_Master}).
  \item Utiliser des sauvegardes physiques (Percona XtraBackup) pour initialiser rapidement les esclaves.
  \item Mettre en place une supervision (Zabbix, Prometheus, etc.) pour détecter les pannes.
\end{itemize}
\end{tcolorbox}


\end{document}
